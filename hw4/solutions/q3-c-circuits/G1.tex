\begin{center}
\begin{circuitikz}[line width=.7pt]
  \draw (0,0) node[and port] (and1) {};
\draw (and1.in 1) -- ++(-0.5,0) node[label=left:a] {};
\draw (and1.in 2) -- ++(-0.5,0) node[label=left:b] {};

\draw (and1.out) -- ++(1,0) node[nor port, anchor=in 1] (nor1) {};
\draw (nor1) ++(0,-2) node[and port] (and2) {};

\draw (and1.in 1) ++(-0.5,-1.5) node[label=left:c] (c) {};
\draw (c) -| (nor1.in 2);
\draw (c) -| (and2.in 1);

\draw (c |- and2.in 2) node[label=left:d] {} -- (and2.in 2);

\draw (nor1.out) -| ++(2,-1) node[or port, anchor=in 1] (or2) {};
\draw (and2.out) -| ++(0.3,-0.5) node[or port, anchor=in 1] (or1) {};
\draw (or1.in 2) -- ++(0,0) node[label=left:e] {};
\draw (or1.out) -| (or2.in 2);

\draw (or2.out) -- ++(0,0) node[label=right:m] {};

\draw (nor1.in 1) node[label=f] {};
\draw (nor1.in 2) node[label=left:g] {};
\draw (and2.in 1) node[label=left:h] {};
\draw (or1.in 1) node[label=left:i] {};
\draw (or2.in 1) node[label=left:j] {};
\draw (or2.in 2) node[label=left:k] {};

  % G1 in 1
  \draw[color=red, -latexslim] (and1.in 1) ++(0,0.25) -- ++(0,-0.5);
  \draw[color=red, -latexslim] (and1.in 1) ++(0.1,-0.25) -- ++(0,0.5);
  % G1 in 2
  \draw[-latexslim] (and1.in 2) ++(0,0.25) -- ++(0,-0.5);
  \draw[color=red, -latexslim] (and1.in 2) ++(0.1,-0.25) -- ++(0,0.5);
  % c
  \draw[color=red, -latexslim] (and1.in 1) ++(0,-1.25) -- ++(0,-0.5);
  \draw[color=red, -latexslim] (and1.in 1) ++(0.1,-1.75) -- ++(0,0.5);
  % G2 in 1
  \draw[-latexslim] (nor1.in 1) ++(0,0.25) -- ++(0,-0.5);
  \draw[-latexslim] (nor1.in 1) ++(0.1,-0.25) -- ++(0,0.5);
  % G2 in 2
  \draw[color=red, -latexslim] (nor1.in 2) ++(0,0.25) -- ++(0,-0.5);
  \draw[-latexslim] (nor1.in 2) ++(0.1,-0.25) -- ++(0,0.5);
  % G3 in 1
  \draw[-latexslim] (and2.in 1) ++(0,0.25) -- ++(0,-0.5);
  \draw[color=red, -latexslim] (and2.in 1) ++(0.1,-0.25) -- ++(0,0.5);
  % G3 in 2
  \draw[color=red, -latexslim] (and2.in 2) ++(0,0.25) -- ++(0,-0.5);
  \draw[color=red, -latexslim] (and2.in 2) ++(0.1,-0.25) -- ++(0,0.5);
  % G4 in 1
  \draw[-latexslim] (or1.in 1) ++(0,0.25) -- ++(0,-0.5);
  \draw[-latexslim] (or1.in 1) ++(0.1,-0.25) -- ++(0,0.5);
  % G4 in 2
  \draw[color=red, -latexslim] (or1.in 2) ++(0,0.25) -- ++(0,-0.5);
  \draw[-latexslim] (or1.in 2) ++(0.1,-0.25) -- ++(0,0.5);
  % G5 in 1
  \draw[-latexslim] (or2.in 1) ++(0,0.25) -- ++(0,-0.5);
  \draw[-latexslim] (or2.in 1) ++(0.1,-0.25) -- ++(0,0.5);
  % G5 in 2
  \draw[-latexslim] (or2.in 2) ++(0,0.25) -- ++(0,-0.5);
  \draw[-latexslim] (or2.in 2) ++(0.1,-0.25) -- ++(0,0.5);
  % G5 out
  \draw[-latexslim] (or2.out) ++(0,0.25) -- ++(0,-0.5);
  \draw[-latexslim] (or2.out) ++(0.1,-0.25) -- ++(0,0.5);

  \draw (and1) node[left=5pt] {G1};
  \draw (nor1) node[left=5pt] {G2};
  \draw (and2) node[left=5pt] {G3};
  \draw (or1) node[left=5pt] {G4};
  \draw (or2) node[left=5pt] {G5};
\end{circuitikz}
\captionof{figure}{At G1, f-sa1 dominates a-sa1 and b-sa1. b-sa0 and f-sa0 are equivalent to a-sa0.}
\end{center}
