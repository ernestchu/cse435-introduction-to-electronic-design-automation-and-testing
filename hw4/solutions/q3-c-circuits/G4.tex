\begin{center}
\begin{circuitikz}[line width=.7pt]
  \draw (0,0) node[nor port] (G1) {};
\draw (G1.out) -- ++(0,0) node[nand port, anchor=in 1] (G2) {};
\draw (G2.in 2) -- ++(0,-1) node[xnor port, anchor=in 1] (G3) {};
\draw (G2.out) |- ++(0.5,-0.5) node[nor port, anchor=in 1] (G4) {};
\draw (G3.out) |- (G4.in 2);

\draw (G1) node[left=6pt] {G1};
\draw (G2) node[left=6pt] {G2};
\draw (G3) node[left=6pt] {G3};
\draw (G4) node[left=6pt] {G4};

  % G1 in 1
  \draw[color=red, -latexslim] (and1.in 1) ++(0,0.25) -- ++(0,-0.5);
  \draw[color=red, -latexslim] (and1.in 1) ++(0.1,-0.25) -- ++(0,0.5);
  % G1 in 2
  \draw[color=red, -latexslim] (and1.in 2) ++(0,0.25) -- ++(0,-0.5);
  \draw[color=red, -latexslim] (and1.in 2) ++(0.1,-0.25) -- ++(0,0.5);
  % c
  \draw[color=red, -latexslim] (and1.in 1) ++(0,-1.25) -- ++(0,-0.5);
  \draw[color=red, -latexslim] (and1.in 1) ++(0.1,-1.75) -- ++(0,0.5);
  % G2 in 1
  \draw[color=red, -latexslim] (nor1.in 1) ++(0,0.25) -- ++(0,-0.5);
  \draw[color=red, -latexslim] (nor1.in 1) ++(0.1,-0.25) -- ++(0,0.5);
  % G2 in 2
  \draw[color=red, -latexslim] (nor1.in 2) ++(0,0.25) -- ++(0,-0.5);
  \draw[color=red, -latexslim] (nor1.in 2) ++(0.1,-0.25) -- ++(0,0.5);
  % G3 in 1
  \draw[color=red, -latexslim] (and2.in 1) ++(0,0.25) -- ++(0,-0.5);
  \draw[color=red, -latexslim] (and2.in 1) ++(0.1,-0.25) -- ++(0,0.5);
  % G3 in 2
  \draw[color=red, -latexslim] (and2.in 2) ++(0,0.25) -- ++(0,-0.5);
  \draw[color=red, -latexslim] (and2.in 2) ++(0.1,-0.25) -- ++(0,0.5);
  % G4 in 1
  \draw[color=red, -latexslim] (or1.in 1) ++(0,0.25) -- ++(0,-0.5);
  \draw[color=red, -latexslim] (or1.in 1) ++(0.1,-0.25) -- ++(0,0.5);
  % G4 in 2
  \draw[color=red, -latexslim] (or1.in 2) ++(0,0.25) -- ++(0,-0.5);
  \draw[-latexslim] (or1.in 2) ++(0.1,-0.25) -- ++(0,0.5);
  % G5 in 1
  \draw[color=red, -latexslim] (or2.in 1) ++(0,0.25) -- ++(0,-0.5);
  \draw[color=red, -latexslim] (or2.in 1) ++(0.1,-0.25) -- ++(0,0.5);
  % G5 in 2
  \draw[-latexslim] (or2.in 2) ++(0,0.25) -- ++(0,-0.5);
  \draw[-latexslim] (or2.in 2) ++(0.1,-0.25) -- ++(0,0.5);
  % G5 out
  \draw[-latexslim] (or2.out) ++(0,0.25) -- ++(0,-0.5);
  \draw[-latexslim] (or2.out) ++(0.1,-0.25) -- ++(0,0.5);

  \draw (and1) node[left=5pt] {G1};
  \draw (nor1) node[left=5pt] {G2};
  \draw (and2) node[left=5pt] {G3};
  \draw (or1) node[left=5pt] {G4};
  \draw (or2) node[left=5pt] {G5};
\end{circuitikz}
\captionof{figure}{At G4, k-sa0 dominates i-sa0 and e-sa0. e-sa1 is equivalent to i-sa1.}
\end{center}
