In order to find the stuck-at-0 fault at F, G must be non-dominant, which is $1$, and F must be $1$. G's value of $1$ indicates $\overline{A \cdot B} = 1$; F's value of $1$ indicates $\overline{B \cdot C} = 1$. (A, B, C) can be either ($\times$, 0, $\times$) or (0, 1, 0), where $\times$ denotes the don't-care condition.

F(sa0) is equivalent to G(sa0) and H(sa1). With multiplicity = 2, the identical test pattern can detect

\begin{itemize}
\item F(sa0) and G(sa0)
\item F(sa0) and H(sa1)
\item G(sa0) and H(sa1)
\end{itemize}

