Primary inputs and fanout branches form a sufficient set of checkpoints in an irredundant combinational circuit. Therefore, there are 7 checkpoints in this circuit, as shown in Figure 2.
\begin{center}
\begin{circuitikz}[line width=.7pt]
  \draw (0,0) node[nor port] (G1) {};
\draw (G1.out) -- ++(0,0) node[nand port, anchor=in 1] (G2) {};
\draw (G2.in 2) -- ++(0,-1) node[xnor port, anchor=in 1] (G3) {};
\draw (G2.out) |- ++(0.5,-0.5) node[nor port, anchor=in 1] (G4) {};
\draw (G3.out) |- (G4.in 2);

\draw (G1) node[left=6pt] {G1};
\draw (G2) node[left=6pt] {G2};
\draw (G3) node[left=6pt] {G3};
\draw (G4) node[left=6pt] {G4};

  \draw[color=red] (and1.in 1) ++(0,0) to[short, *-] ++(0,0);
  \draw[color=red] (and1.in 2) ++(0,0) to[short, *-] ++(0,0);
  \draw[color=red] (nor1.in 2) ++(0,0) to[short, *-] ++(0,0);
  \draw[color=red] (and2.in 1) ++(0,0) to[short, *-] ++(0,0);
  \draw[color=red] (and2.in 2) ++(0,0) to[short, *-] ++(0,0);
  \draw[color=red] (or1.in 2) ++(0,0) to[short, *-] ++(0,0);
  \draw[color=red] (and1.in 1) ++(0,-1.5) to[short, *-] ++(0,0);
\end{circuitikz}
\captionof{figure}{}
\end{center}

These 7 checkpoints could produce 2 $\times$ 7 = 14 SSA faults.

\begin{center}
  \begin{tabular}{lll}
    \specialrule{.1em}{.05em}{.05em} 
    Gate & Equivalence Set(s) &	Dominant Set(s) \\
    \hline
    OR & \{a-sa1, b-sa1, out-sa1\} & \{out-sa0: a-sa0, b-sa0\} \\
    NOR & \{a-sa1, b-sa1, out-sa0\} & \{out-sa1: a-sa0, b-sa0\} \\
    AND & \{a-sa0, b-sa0, out-sa0\} & \{out-sa1: a-sa1, b-sa1\} \\
    NAND & \{a-sa0, b-sa0, out-sa1\} & \{out-sa0: a-sa1, b-sa1\} \\
    Buffer & \vtop{\hbox{\strut \{in-sa0, out-sa0\}}\hbox{\strut \{in-sa1, out-sa1\}}} & null \\
    NOT & \vtop{\hbox{\strut \{in-sa1, out-sa0\}}\hbox{\strut \{in-sa0, out-sa1\}}} & null \\
    \specialrule{.1em}{.05em}{.05em} 
  \end{tabular}
\end{center}

% With fault dominance, s-a-1 at \textbf{a} and s-a-1 at \textbf{b} are dominated by s-a-1 at \textbf{f}, and we can replace the former two with the latter. Likewise, s-a-1 at \textbf{h} and s-a-1 at \textbf{d} are dominated by s-a-1 at \textbf{i}. As a result, the number of faults is reduced by 2, leaving 10 SSA faults after applying checkpoint theorem and fault dominance.

