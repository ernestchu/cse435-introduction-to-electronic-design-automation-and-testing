Primary inputs and fanout branches form a sufficient set of checkpoints in an irredundant combinational circuit. Therefore, there are 7 checkpoints in this circuit, as shown in Figure 2.
\begin{center}
\begin{circuitikz}[line width=.7pt]
  \draw (0,0) node[and port] (and1) {};
\draw (and1.in 1) -- ++(-0.5,0) node[label=left:a] {};
\draw (and1.in 2) -- ++(-0.5,0) node[label=left:b] {};

\draw (and1.out) -- ++(1,0) node[nor port, anchor=in 1] (nor1) {};
\draw (nor1) ++(0,-2) node[and port] (and2) {};

\draw (and1.in 1) ++(-0.5,-1.5) node[label=left:c] (c) {};
\draw (c) -| (nor1.in 2);
\draw (c) -| (and2.in 1);

\draw (c |- and2.in 2) node[label=left:d] {} -- (and2.in 2);

\draw (nor1.out) -| ++(2,-1) node[or port, anchor=in 1] (or2) {};
\draw (and2.out) -| ++(0.3,-0.5) node[or port, anchor=in 1] (or1) {};
\draw (or1.in 2) -- ++(0,0) node[label=left:e] {};
\draw (or1.out) -| (or2.in 2);

\draw (or2.out) -- ++(0,0) node[label=right:m] {};

\draw (nor1.in 1) node[label=f] {};
\draw (nor1.in 2) node[label=left:g] {};
\draw (and2.in 1) node[label=left:h] {};
\draw (or1.in 1) node[label=left:i] {};
\draw (or2.in 1) node[label=left:j] {};
\draw (or2.in 2) node[label=left:k] {};

  \draw[color=red] (and1.in 1) ++(0,0) to[short, *-] ++(0,0);
  \draw[color=red] (and1.in 2) ++(0,0) to[short, *-] ++(0,0);
  \draw[color=red] (nor1.in 2) ++(0,0) to[short, *-] ++(0,0);
  \draw[color=red] (and2.in 1) ++(0,0) to[short, *-] ++(0,0);
  \draw[color=red] (and2.in 2) ++(0,0) to[short, *-] ++(0,0);
  \draw[color=red] (or1.in 2) ++(0,0) to[short, *-] ++(0,0);
  \draw[color=red] (and1.in 1) ++(0,-1.5) to[short, *-] ++(0,0);
\end{circuitikz}
\captionof{figure}{}
\end{center}

These 7 checkpoints can produce 2 $\times$ 7 = 14 SSA faults. To further reduce the number of faults, we can use fault collapsing method, including fault equivalence and fault dominance.

\begin{center}
\begin{circuitikz}[line width=.7pt]
  \draw (0,0) node[or port] (or) {};
  \draw (or.in 1) -- ++(0,0) node[label=left:a] {};
  \draw (or.in 2) -- ++(0,0) node[label=left:b] {};
  \draw (or.out) -- ++(0,0) node[label=right:out] {};
\end{circuitikz}
\captionof{figure}{}
\end{center}

\paragraph{Fault equivalence}
For OR gate shown in Figure 3 as example, to test the stuck-at-1 (sa1) fault at a, denoted as a-sa1, we must force logic-0 at a to activate the fault, and we also need the other input (b) as non-controlling value, which is 0 for OR gate. The test vector is thus (0, 0), which is also the test vector for sa1 at b, denoted as b-sa1. To test sa1 at output, we only need to make all the inputs of OR gate logic-0 to activate the fault. We can observe that all the test vectors for a-sa1, b-sa1 and out-sa1 are identical, so the OR gate has an Equivalence Set of {a-sa1, b-sa1, out-sa1}, indicating we can only keep one fault and collapse others.

\paragraph{Fault dominance}
For OR gate shown in Figure 3 as example, to test the stuck-at-0 (sa0) fault at a, denoted as a-sa0, we must force logic-1 at a to activate the fault, and we also need the other input (b) as non-controlling value, which is 0 for OR gate. The test vector is thus (1, 0). Similarly, the test vector for sa0 at b, denoted as b-sa0, is (0, 1). To test sa0 at output, we need to make one of the inputs of OR gate logic-1 to activate the fault. The test vector is thus (1, $\times$) or ($\times$, 1). We can observe that the test vector for out-sa0 covers the ones for a-sa0 and b-sa0, i.e., out-sa0 dominates a-sa0 and b-sa0, so the OR gate has a Dominance Set of {out-sa0: a-sa0, b-sa0}, indicating we can collapse the dominator since the test inputs for the others are sufficient.

Table 1 shows the equivalence sets and dominance sets for common gates. \\

\begin{center}
  \def\arraystretch{1.2}
  \begin{tabular}{lll}
    \specialrule{.1em}{.05em}{.05em} 
    Gate & Equivalence Set(s) &	Dominance Set(s) \\
    \hline
    OR & \{a-sa1, b-sa1, out-sa1\} & \{out-sa0: a-sa0, b-sa0\} \\
    NOR & \{a-sa1, b-sa1, out-sa0\} & \{out-sa1: a-sa0, b-sa0\} \\
    AND & \{a-sa0, b-sa0, out-sa0\} & \{out-sa1: a-sa1, b-sa1\} \\
    NAND & \{a-sa0, b-sa0, out-sa1\} & \{out-sa0: a-sa1, b-sa1\} \\
    NOT & \vtop{\hbox{\strut \{in-sa1, out-sa0\}}\hbox{\strut \{in-sa0, out-sa1\}}} & null \\
    Buffer & \vtop{\hbox{\strut \{in-sa0, out-sa0\}}\hbox{\strut \{in-sa1, out-sa1\}}} & null \\
    \specialrule{.1em}{.05em}{.05em} 
  \end{tabular}
  \captionof{table}{The equivalence sets and dominance sets for common gates.}
\end{center}

From Figure 4 to Figure 8 below, the red arrows indicate the necessary stuck-at faults, while the black ones indicates the collapsed faults after each steps.

\begin{center}
\begin{circuitikz}[line width=.7pt]
  \draw (0,0) node[and port] (and1) {};
\draw (and1.in 1) -- ++(-0.5,0) node[label=left:a] {};
\draw (and1.in 2) -- ++(-0.5,0) node[label=left:b] {};

\draw (and1.out) -- ++(1,0) node[nor port, anchor=in 1] (nor1) {};
\draw (nor1) ++(0,-2) node[and port] (and2) {};

\draw (and1.in 1) ++(-0.5,-1.5) node[label=left:c] (c) {};
\draw (c) -| (nor1.in 2);
\draw (c) -| (and2.in 1);

\draw (c |- and2.in 2) node[label=left:d] {} -- (and2.in 2);

\draw (nor1.out) -| ++(2,-1) node[or port, anchor=in 1] (or2) {};
\draw (and2.out) -| ++(0.3,-0.5) node[or port, anchor=in 1] (or1) {};
\draw (or1.in 2) -- ++(0,0) node[label=left:e] {};
\draw (or1.out) -| (or2.in 2);

\draw (or2.out) -- ++(0,0) node[label=right:m] {};

\draw (nor1.in 1) node[label=f] {};
\draw (nor1.in 2) node[label=left:g] {};
\draw (and2.in 1) node[label=left:h] {};
\draw (or1.in 1) node[label=left:i] {};
\draw (or2.in 1) node[label=left:j] {};
\draw (or2.in 2) node[label=left:k] {};

  % G1 in 1
  \draw[color=red, -latexslim] (and1.in 1) ++(0,0.25) -- ++(0,-0.5);
  \draw[color=red, -latexslim] (and1.in 1) ++(0.1,-0.25) -- ++(0,0.5);
  % G1 in 2
  \draw[color=red, -latexslim] (and1.in 2) ++(0,0.25) -- ++(0,-0.5);
  \draw[color=red, -latexslim] (and1.in 2) ++(0.1,-0.25) -- ++(0,0.5);
  % c
  \draw[color=red, -latexslim] (and1.in 1) ++(0,-1.25) -- ++(0,-0.5);
  \draw[color=red, -latexslim] (and1.in 1) ++(0.1,-1.75) -- ++(0,0.5);
  % G2 in 1
  \draw[color=red, -latexslim] (nor1.in 1) ++(0,0.25) -- ++(0,-0.5);
  \draw[color=red, -latexslim] (nor1.in 1) ++(0.1,-0.25) -- ++(0,0.5);
  % G2 in 2
  \draw[color=red, -latexslim] (nor1.in 2) ++(0,0.25) -- ++(0,-0.5);
  \draw[color=red, -latexslim] (nor1.in 2) ++(0.1,-0.25) -- ++(0,0.5);
  % G3 in 1
  \draw[color=red, -latexslim] (and2.in 1) ++(0,0.25) -- ++(0,-0.5);
  \draw[color=red, -latexslim] (and2.in 1) ++(0.1,-0.25) -- ++(0,0.5);
  % G3 in 2
  \draw[color=red, -latexslim] (and2.in 2) ++(0,0.25) -- ++(0,-0.5);
  \draw[color=red, -latexslim] (and2.in 2) ++(0.1,-0.25) -- ++(0,0.5);
  % G4 in 1
  \draw[color=red, -latexslim] (or1.in 1) ++(0,0.25) -- ++(0,-0.5);
  \draw[color=red, -latexslim] (or1.in 1) ++(0.1,-0.25) -- ++(0,0.5);
  % G4 in 2
  \draw[color=red, -latexslim] (or1.in 2) ++(0,0.25) -- ++(0,-0.5);
  \draw[color=red, -latexslim] (or1.in 2) ++(0.1,-0.25) -- ++(0,0.5);
  % G5 in 1
  \draw[color=red, -latexslim] (or2.in 1) ++(0,0.25) -- ++(0,-0.5);
  \draw[color=red, -latexslim] (or2.in 1) ++(0.1,-0.25) -- ++(0,0.5);
  % G5 in 2
  \draw[color=red, -latexslim] (or2.in 2) ++(0,0.25) -- ++(0,-0.5);
  \draw[color=red, -latexslim] (or2.in 2) ++(0.1,-0.25) -- ++(0,0.5);
  % G5 out
  \draw[color=red, -latexslim] (or2.out) ++(0,0.25) -- ++(0,-0.5);
  \draw[color=red, -latexslim] (or2.out) ++(0.1,-0.25) -- ++(0,0.5);

  \draw (and1) node[left=5pt] {G1};
  \draw (nor1) node[left=5pt] {G2};
  \draw (and2) node[left=5pt] {G3};
  \draw (or1) node[left=5pt] {G4};
  \draw (or2) node[left=5pt] {G5};
\end{circuitikz}
\captionof{figure}{Initial 24 stuck-at faults.}
\end{center}

\begin{center}
\begin{circuitikz}[line width=.7pt]
  \draw (0,0) node[and port] (and1) {};
\draw (and1.in 1) -- ++(-0.5,0) node[label=left:a] {};
\draw (and1.in 2) -- ++(-0.5,0) node[label=left:b] {};

\draw (and1.out) -- ++(1,0) node[nor port, anchor=in 1] (nor1) {};
\draw (nor1) ++(0,-2) node[and port] (and2) {};

\draw (and1.in 1) ++(-0.5,-1.5) node[label=left:c] (c) {};
\draw (c) -| (nor1.in 2);
\draw (c) -| (and2.in 1);

\draw (c |- and2.in 2) node[label=left:d] {} -- (and2.in 2);

\draw (nor1.out) -| ++(2,-1) node[or port, anchor=in 1] (or2) {};
\draw (and2.out) -| ++(0.3,-0.5) node[or port, anchor=in 1] (or1) {};
\draw (or1.in 2) -- ++(0,0) node[label=left:e] {};
\draw (or1.out) -| (or2.in 2);

\draw (or2.out) -- ++(0,0) node[label=right:m] {};

\draw (nor1.in 1) node[label=f] {};
\draw (nor1.in 2) node[label=left:g] {};
\draw (and2.in 1) node[label=left:h] {};
\draw (or1.in 1) node[label=left:i] {};
\draw (or2.in 1) node[label=left:j] {};
\draw (or2.in 2) node[label=left:k] {};

  \draw[color=red, -latexslim] (and1.in 1) ++(0,0.25) -- ++(0,-0.5);
  \draw[color=red, -latexslim] (and1.in 1) ++(0.1,-0.25) -- ++(0,0.5);

  \draw[color=red, -latexslim] (and1.in 2) ++(0,0.25) -- ++(0,-0.5);
  \draw[color=red, -latexslim] (and1.in 2) ++(0.1,-0.25) -- ++(0,0.5);

  \draw[color=red, -latexslim] (and1.in 1) ++(0,-1.25) -- ++(0,-0.5);
  \draw[color=red, -latexslim] (and1.in 1) ++(0.1,-1.75) -- ++(0,0.5);

  \draw[color=red, -latexslim] (nor1.in 2) ++(0,0.25) -- ++(0,-0.5);
  \draw[color=red, -latexslim] (nor1.in 2) ++(0.1,-0.25) -- ++(0,0.5);

  \draw[color=red, -latexslim] (and2.in 1) ++(0,0.25) -- ++(0,-0.5);
  \draw[color=red, -latexslim] (and2.in 1) ++(0.1,-0.25) -- ++(0,0.5);

  \draw[color=red, -latexslim] (and2.in 2) ++(0,0.25) -- ++(0,-0.5);
  \draw[color=red, -latexslim] (and2.in 2) ++(0.1,-0.25) -- ++(0,0.5);

  \draw[color=red, -latexslim] (or1.in 2) ++(0,0.25) -- ++(0,-0.5);
  \draw[-latexslim] (or1.in 2) ++(0.1,-0.25) -- ++(0,0.5);
  \draw[color=red, -latexslim] (or1.in 1) ++(0.1,-0.25) -- ++(0,0.5);

  \draw (and1) node[left=5pt] {G1};
  \draw (nor1) node[left=5pt] {G2};
  \draw (and2) node[left=5pt] {G3};
  \draw (or1) node[left=5pt] {G4};
  \draw (or2) node[left=5pt] {G5};
\end{circuitikz}
\captionof{figure}{At G4, e-sa1 is equivalent to i-sa1.}
\end{center}

\begin{center}
\begin{circuitikz}[line width=.7pt]
  \draw (0,0) node[and port] (and1) {};
\draw (and1.in 1) -- ++(-0.5,0) node[label=left:a] {};
\draw (and1.in 2) -- ++(-0.5,0) node[label=left:b] {};

\draw (and1.out) -- ++(1,0) node[nor port, anchor=in 1] (nor1) {};
\draw (nor1) ++(0,-2) node[and port] (and2) {};

\draw (and1.in 1) ++(-0.5,-1.5) node[label=left:c] (c) {};
\draw (c) -| (nor1.in 2);
\draw (c) -| (and2.in 1);

\draw (c |- and2.in 2) node[label=left:d] {} -- (and2.in 2);

\draw (nor1.out) -| ++(2,-1) node[or port, anchor=in 1] (or2) {};
\draw (and2.out) -| ++(0.3,-0.5) node[or port, anchor=in 1] (or1) {};
\draw (or1.in 2) -- ++(0,0) node[label=left:e] {};
\draw (or1.out) -| (or2.in 2);

\draw (or2.out) -- ++(0,0) node[label=right:m] {};

\draw (nor1.in 1) node[label=f] {};
\draw (nor1.in 2) node[label=left:g] {};
\draw (and2.in 1) node[label=left:h] {};
\draw (or1.in 1) node[label=left:i] {};
\draw (or2.in 1) node[label=left:j] {};
\draw (or2.in 2) node[label=left:k] {};

  \draw[color=red, -latexslim] (and1.in 1) ++(0,0.25) -- ++(0,-0.5);
  \draw[color=red, -latexslim] (and1.in 1) ++(0.1,-0.25) -- ++(0,0.5);

  \draw[color=red, -latexslim] (and1.in 2) ++(0,0.25) -- ++(0,-0.5);
  \draw[color=red, -latexslim] (and1.in 2) ++(0.1,-0.25) -- ++(0,0.5);

  \draw[color=red, -latexslim] (and1.in 1) ++(0,-1.25) -- ++(0,-0.5);
  \draw[color=red, -latexslim] (and1.in 1) ++(0.1,-1.75) -- ++(0,0.5);

  \draw[color=red, -latexslim] (nor1.in 2) ++(0,0.25) -- ++(0,-0.5);
  \draw[color=red, -latexslim] (nor1.in 2) ++(0.1,-0.25) -- ++(0,0.5);

  \draw[-latexslim] (and2.in 1) ++(0,0.25) -- ++(0,-0.5);
  \draw[color=red, -latexslim] (and2.in 1) ++(0.1,-0.25) -- ++(0,0.5);

  \draw[color=red, -latexslim] (and2.in 2) ++(0,0.25) -- ++(0,-0.5);
  \draw[color=red, -latexslim] (and2.in 2) ++(0.1,-0.25) -- ++(0,0.5);

  \draw[color=red, -latexslim] (or1.in 2) ++(0,0.25) -- ++(0,-0.5);
  \draw[-latexslim] (or1.in 2) ++(0.1,-0.25) -- ++(0,0.5);
  \draw[-latexslim] (or1.in 1) ++(0.1,-0.25) -- ++(0,0.5);

  \draw (and1) node[left=5pt] {G1};
  \draw (nor1) node[left=5pt] {G2};
  \draw (and2) node[left=5pt] {G3};
  \draw (or1) node[left=5pt] {G4};
  \draw (or2) node[left=5pt] {G5};
\end{circuitikz}
\captionof{figure}{At G3, i-sa1 dominates h-sa1 and d-sa1. h-sa0 is equivalent to d-sa0. }
\end{center}

\begin{center}
\begin{circuitikz}[line width=.7pt]
  \draw (0,0) node[and port] (and1) {};
\draw (and1.in 1) -- ++(-0.5,0) node[label=left:a] {};
\draw (and1.in 2) -- ++(-0.5,0) node[label=left:b] {};

\draw (and1.out) -- ++(1,0) node[nor port, anchor=in 1] (nor1) {};
\draw (nor1) ++(0,-2) node[and port] (and2) {};

\draw (and1.in 1) ++(-0.5,-1.5) node[label=left:c] (c) {};
\draw (c) -| (nor1.in 2);
\draw (c) -| (and2.in 1);

\draw (c |- and2.in 2) node[label=left:d] {} -- (and2.in 2);

\draw (nor1.out) -| ++(2,-1) node[or port, anchor=in 1] (or2) {};
\draw (and2.out) -| ++(0.3,-0.5) node[or port, anchor=in 1] (or1) {};
\draw (or1.in 2) -- ++(0,0) node[label=left:e] {};
\draw (or1.out) -| (or2.in 2);

\draw (or2.out) -- ++(0,0) node[label=right:m] {};

\draw (nor1.in 1) node[label=f] {};
\draw (nor1.in 2) node[label=left:g] {};
\draw (and2.in 1) node[label=left:h] {};
\draw (or1.in 1) node[label=left:i] {};
\draw (or2.in 1) node[label=left:j] {};
\draw (or2.in 2) node[label=left:k] {};

  % G1 in 1
  \draw[color=red, -latexslim] (and1.in 1) ++(0,0.25) -- ++(0,-0.5);
  \draw[color=red, -latexslim] (and1.in 1) ++(0.1,-0.25) -- ++(0,0.5);
  % G1 in 2
  \draw[color=red, -latexslim] (and1.in 2) ++(0,0.25) -- ++(0,-0.5);
  \draw[color=red, -latexslim] (and1.in 2) ++(0.1,-0.25) -- ++(0,0.5);
  % c
  \draw[color=red, -latexslim] (and1.in 1) ++(0,-1.25) -- ++(0,-0.5);
  \draw[color=red, -latexslim] (and1.in 1) ++(0.1,-1.75) -- ++(0,0.5);
  % G2 in 1
  \draw[color=red, -latexslim] (nor1.in 1) ++(0,0.25) -- ++(0,-0.5);
  \draw[color=red, -latexslim] (nor1.in 1) ++(0.1,-0.25) -- ++(0,0.5);
  % G2 in 2
  \draw[color=red, -latexslim] (nor1.in 2) ++(0,0.25) -- ++(0,-0.5);
  \draw[-latexslim] (nor1.in 2) ++(0.1,-0.25) -- ++(0,0.5);
  % G3 in 1
  \draw[-latexslim] (and2.in 1) ++(0,0.25) -- ++(0,-0.5);
  \draw[color=red, -latexslim] (and2.in 1) ++(0.1,-0.25) -- ++(0,0.5);
  % G3 in 2
  \draw[color=red, -latexslim] (and2.in 2) ++(0,0.25) -- ++(0,-0.5);
  \draw[color=red, -latexslim] (and2.in 2) ++(0.1,-0.25) -- ++(0,0.5);
  % G4 in 1
  \draw[-latexslim] (or1.in 1) ++(0,0.25) -- ++(0,-0.5);
  \draw[-latexslim] (or1.in 1) ++(0.1,-0.25) -- ++(0,0.5);
  % G4 in 2
  \draw[color=red, -latexslim] (or1.in 2) ++(0,0.25) -- ++(0,-0.5);
  \draw[-latexslim] (or1.in 2) ++(0.1,-0.25) -- ++(0,0.5);
  % G5 in 1
  \draw[-latexslim] (or2.in 1) ++(0,0.25) -- ++(0,-0.5);
  \draw[-latexslim] (or2.in 1) ++(0.1,-0.25) -- ++(0,0.5);
  % G5 in 2
  \draw[-latexslim] (or2.in 2) ++(0,0.25) -- ++(0,-0.5);
  \draw[-latexslim] (or2.in 2) ++(0.1,-0.25) -- ++(0,0.5);
  % G5 out
  \draw[-latexslim] (or2.out) ++(0,0.25) -- ++(0,-0.5);
  \draw[-latexslim] (or2.out) ++(0.1,-0.25) -- ++(0,0.5);

  \draw (and1) node[left=5pt] {G1};
  \draw (nor1) node[left=5pt] {G2};
  \draw (and2) node[left=5pt] {G3};
  \draw (or1) node[left=5pt] {G4};
  \draw (or2) node[left=5pt] {G5};
\end{circuitikz}
\captionof{figure}{At G2, j-sa1 dominates f-sa0 and g-sa0. g-sa1 and j-sa0 are equivalent to f-sa1.}
\end{center}

\begin{center}
\begin{circuitikz}[line width=.7pt]
  \draw (0,0) node[and port] (and1) {};
\draw (and1.in 1) -- ++(-0.5,0) node[label=left:a] {};
\draw (and1.in 2) -- ++(-0.5,0) node[label=left:b] {};

\draw (and1.out) -- ++(1,0) node[nor port, anchor=in 1] (nor1) {};
\draw (nor1) ++(0,-2) node[and port] (and2) {};

\draw (and1.in 1) ++(-0.5,-1.5) node[label=left:c] (c) {};
\draw (c) -| (nor1.in 2);
\draw (c) -| (and2.in 1);

\draw (c |- and2.in 2) node[label=left:d] {} -- (and2.in 2);

\draw (nor1.out) -| ++(2,-1) node[or port, anchor=in 1] (or2) {};
\draw (and2.out) -| ++(0.3,-0.5) node[or port, anchor=in 1] (or1) {};
\draw (or1.in 2) -- ++(0,0) node[label=left:e] {};
\draw (or1.out) -| (or2.in 2);

\draw (or2.out) -- ++(0,0) node[label=right:m] {};

\draw (nor1.in 1) node[label=f] {};
\draw (nor1.in 2) node[label=left:g] {};
\draw (and2.in 1) node[label=left:h] {};
\draw (or1.in 1) node[label=left:i] {};
\draw (or2.in 1) node[label=left:j] {};
\draw (or2.in 2) node[label=left:k] {};

  % G1 in 1
  \draw[color=red, -latexslim] (and1.in 1) ++(0,0.25) -- ++(0,-0.5);
  \draw[color=red, -latexslim] (and1.in 1) ++(0.1,-0.25) -- ++(0,0.5);
  % G1 in 2
  \draw[color=red, -latexslim] (and1.in 2) ++(0,0.25) -- ++(0,-0.5);
  \draw[color=red, -latexslim] (and1.in 2) ++(0.1,-0.25) -- ++(0,0.5);
  % c
  \draw[color=red, -latexslim] (and1.in 1) ++(0,-1.25) -- ++(0,-0.5);
  \draw[color=red, -latexslim] (and1.in 1) ++(0.1,-1.75) -- ++(0,0.5);
  % G2 in 1
  \draw[-latexslim] (nor1.in 1) ++(0,0.25) -- ++(0,-0.5);
  \draw[-latexslim] (nor1.in 1) ++(0.1,-0.25) -- ++(0,0.5);
  % G2 in 2
  \draw[color=red, -latexslim] (nor1.in 2) ++(0,0.25) -- ++(0,-0.5);
  \draw[color=red, -latexslim] (nor1.in 2) ++(0.1,-0.25) -- ++(0,0.5);
  % G3 in 1
  \draw[color=red, -latexslim] (and2.in 1) ++(0,0.25) -- ++(0,-0.5);
  \draw[color=red, -latexslim] (and2.in 1) ++(0.1,-0.25) -- ++(0,0.5);
  % G3 in 2
  \draw[color=red, -latexslim] (and2.in 2) ++(0,0.25) -- ++(0,-0.5);
  \draw[color=red, -latexslim] (and2.in 2) ++(0.1,-0.25) -- ++(0,0.5);
  % G4 in 1
  \draw[-latexslim] (or1.in 1) ++(0,0.25) -- ++(0,-0.5);
  \draw[-latexslim] (or1.in 1) ++(0.1,-0.25) -- ++(0,0.5);
  % G4 in 2
  \draw[color=red, -latexslim] (or1.in 2) ++(0,0.25) -- ++(0,-0.5);
  \draw[color=red, -latexslim] (or1.in 2) ++(0.1,-0.25) -- ++(0,0.5);
  % G5 in 1
  \draw[-latexslim] (or2.in 1) ++(0,0.25) -- ++(0,-0.5);
  \draw[-latexslim] (or2.in 1) ++(0.1,-0.25) -- ++(0,0.5);
  % G5 in 2
  \draw[-latexslim] (or2.in 2) ++(0,0.25) -- ++(0,-0.5);
  \draw[-latexslim] (or2.in 2) ++(0.1,-0.25) -- ++(0,0.5);
  % G5 out
  \draw[-latexslim] (or2.out) ++(0,0.25) -- ++(0,-0.5);
  \draw[-latexslim] (or2.out) ++(0.1,-0.25) -- ++(0,0.5);

  \draw (and1) node[left=5pt] {G1};
  \draw (nor1) node[left=5pt] {G2};
  \draw (and2) node[left=5pt] {G3};
  \draw (or1) node[left=5pt] {G4};
  \draw (or2) node[left=5pt] {G5};
\end{circuitikz}
\captionof{figure}{At G1, f-sa1 dominates a-sa1 and b-sa1. f-sa0 is equivalent to a-sa0. We've already collapsed the faults into 7 check points. For the rest of the steps, just follow problem 3(b) above.}
\end{center}


% With fault dominance, s-a-1 at \textbf{a} and s-a-1 at \textbf{b} are dominated by s-a-1 at \textbf{f}, and we can replace the former two with the latter. Likewise, s-a-1 at \textbf{h} and s-a-1 at \textbf{d} are dominated by s-a-1 at \textbf{i}. As a result, the number of faults is reduced by 2, leaving 10 SSA faults after applying checkpoint theorem and fault dominance.

