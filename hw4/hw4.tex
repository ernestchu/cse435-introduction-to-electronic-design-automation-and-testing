\documentclass[12pt,answers]{exam}
\usepackage{fontspec,graphicx,circuitikz,amsmath,caption}
\setmainfont{Times New Roman}

\headheight 8pt \headsep 20pt \footskip 30pt
\textheight 9in \textwidth 6.5in
\oddsidemargin 0in \evensidemargin 0in
\topmargin -.35in

% a handy function to show the position of nodes
% comment out the second \let command to hide the marker
\def\normalcoord(#1){coordinate(#1)}
\def\showcoord(#1){coordinate(#1) node[circle, red, draw, inner sep=1pt,
  pin={[red, overlay, inner sep=0.5pt, font=\tiny, pin distance=0.1cm,
  pin edge={red, overlay}]45:#1}](){}}
\let\coord=\normalcoord
% \let\coord=\showcoord


\begin{document}
\begin{center}
\LARGE CSE435 Introduction to EDA \& Testing - Spring 2022 \\
\Large Homework Assignment \#4 \\
\Large Shao-Hsuan Chu - B073040018 \\
\end{center}
\bigskip

\begin{questions}
  \question (25\%)
  \begin{parts}
    \part (10\%) For state transition fault model, explain why there are M(N-1) faults for a M-transition N-state machine. Similarly explain why there are $\text{N}^\text{M}$-1 multiple state transition faults.
    \begin{solution}
    % 
(1) do not check the controllability of each line, so it don't make intermediate objectives. It just sets the primary input based on the inversion parity of the entire backtracing path.

(2)/(3) utilize the controllability of each line to choose the backtracing path. They thus set the intermediate objectives in order to apply the heuristics. These heuristics reduce the possibility of encountering a conflict in comparison to (1).

    \end{solution}

    \part (10\%) For stuck-at fault model, explain why there are 2K single stuck-at faults. Similarly explain why there are $\text{3}^\text{K}$-1 multiple stuck-at faults.
    \begin{solution}
    % 
(1) do not check the controllability of each line, so it don't make intermediate objectives. It just sets the primary input based on the inversion parity of the entire backtracing path.

(2)/(3) utilize the controllability of each line to choose the backtracing path. They thus set the intermediate objectives in order to apply the heuristics. These heuristics reduce the possibility of encountering a conflict in comparison to (1).

    \end{solution}

    \part (5\%) Please show the similarity and differences of (single, multiple) fault numbers between the state transition fault model and the stuck-at fault model.
    \begin{solution}
    % 
(1) do not check the controllability of each line, so it don't make intermediate objectives. It just sets the primary input based on the inversion parity of the entire backtracing path.

(2)/(3) utilize the controllability of each line to choose the backtracing path. They thus set the intermediate objectives in order to apply the heuristics. These heuristics reduce the possibility of encountering a conflict in comparison to (1).

    \end{solution}

  \end{parts}

  \question (20\%) Prove that for combinational circuits \textbf{faults dominance is a transitive relation}, i.e. if f dominates g and g dominates h, then f dominates h.
  \begin{solution}
  % When $R_C$ is abnormally high, the output would become slow to charge, which leads to the gate-delay fault. The time required to charge the capacitor $\tau$, which can be formulated as $\tau = RC$, where $R$ is the resistance, and $C$ is the capacitance, thus $\tau \propto R$. The delay is increased by 100 times as well.

To test the charging delay fault, we should first clear the output by setting (A, B) to (0, 1), (1, 0), or (1, 1). At the second time frame, we set (A, B) to (0, 0) to charge the circuit and observe the output to test if it produces significant delay.

\textbf{Answer:} At time frame 1, (A, B) = (0, 1), (1, 0), or (1, 1). At time frame 2, (A, B) = (0, 0).

  \end{solution}

  \question (55\%) In the circuit shown in Figure 1,
  The stuck-open fault in the charging unit (pull-up, PU) makes the output never charged to 1 after discharge to 0. For example, at the first time frame, we discharge the circuit by setting (A, B) to (1, 1). Then we switch B to 0. If the transistor M1 functioned normally, it should charge the output since the ground wire is cut-off by an NMOS. If 0 is instead observed at Y, then there's a stuck-open fault at M1.

\textbf{Answer:} At time frame 1, (A, B) = (1, 1). At time frame 2, (A, B) = (1, 0).

  \begin{parts}
    \part (5\%) How many single stuck-at faults needed to be considered initially?
    \begin{solution}
    % 
(1) do not check the controllability of each line, so it don't make intermediate objectives. It just sets the primary input based on the inversion parity of the entire backtracing path.

(2)/(3) utilize the controllability of each line to choose the backtracing path. They thus set the intermediate objectives in order to apply the heuristics. These heuristics reduce the possibility of encountering a conflict in comparison to (1).

    \end{solution}

    \part (25\%) Applying the \textbf{check point theorem (incl. fault dominance)}, how many check point faults needed to be considered?
    \begin{solution}
    % 
(1) do not check the controllability of each line, so it don't make intermediate objectives. It just sets the primary input based on the inversion parity of the entire backtracing path.

(2)/(3) utilize the controllability of each line to choose the backtracing path. They thus set the intermediate objectives in order to apply the heuristics. These heuristics reduce the possibility of encountering a conflict in comparison to (1).

    \end{solution}

    \part (25\%) Using \textbf{fault dominance} and \textbf{fault equivalence} relations to further reduce the number of stuck-at faults? How many remaining faults needed to be considered?
    \begin{solution}
    % 
(1) do not check the controllability of each line, so it don't make intermediate objectives. It just sets the primary input based on the inversion parity of the entire backtracing path.

(2)/(3) utilize the controllability of each line to choose the backtracing path. They thus set the intermediate objectives in order to apply the heuristics. These heuristics reduce the possibility of encountering a conflict in comparison to (1).

    \end{solution}

  \end{parts}
\end{questions}
\end{document}

