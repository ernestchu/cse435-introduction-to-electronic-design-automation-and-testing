\documentclass[12pt,answers]{exam}
\usepackage{fontspec,graphicx,circuitikz,amsmath,caption,xcolor}
\setmainfont{Times New Roman}

\headheight 8pt \headsep 20pt \footskip 30pt
\textheight 9in \textwidth 6.5in
\oddsidemargin 0in \evensidemargin 0in
\topmargin -.35in

% a handy function to show the position of nodes
% comment out the second \let command to hide the marker
\def\normalcoord(#1){coordinate(#1)}
\def\showcoord(#1){coordinate(#1) node[circle, red, draw, inner sep=1pt,
  pin={[red, overlay, inner sep=0.5pt, font=\tiny, pin distance=0.1cm,
  pin edge={red, overlay}]45:#1}](){}}
\let\coord=\normalcoord
% \let\coord=\showcoord


\begin{document}
\begin{center}
\LARGE CSE435 Introduction to EDA \& Testing - Spring 2022 \\
\Large Homework Assignment \#4 \\
\Large Shao-Hsuan Chu - B073040018 \\
\end{center}
\bigskip

\begin{questions}
  \question (25\%)
  \begin{parts}
    \part (10\%) For state transition fault model, explain why there are M(N-1) faults for a M-transition N-state machine. Similarly explain why there are $\text{N}^\text{M}$-1 multiple state transition faults.
    \begin{solution}
    Denote M and N as the number of transitions and states, respectively. 
\paragraph{Single-state-transition (SST) fault}
For each transition, the possible destination could be either one of the states, so there are (N-1) faulty cases and 1 faultless case. The SST could occur at either one of the transitions, producing M(N-1) distinguishable faults.
\paragraph{Multiple-state-transition (MST) fault}
For each transition, the possible destination could be either one of the states, so there are N cases. Since each transition is independent to other transitions, there are $\text{N}^\text{M}$ distinguishable state-transition diagrams. Disregarding 1 faultless diagram, the number of faults in MST is thus $\text{N}^\text{M}$-1.

    \end{solution}

    \part (10\%) For stuck-at fault model, explain why there are 2K single stuck-at faults. Similarly explain why there are $\text{3}^\text{K}$-1 multiple stuck-at faults.
    \begin{solution}
    Denote K as the number of lines in the circuit.

\paragraph{Single-stuck-at (SSA) fault}
The SSA fault would occur at either one of the lines, and in the SSA faults, there will be the case of stuck-at-0 and stuck-at-1. So the number of distinguishable SSA faults would be 2K.
\paragraph{Multiple-stuck-at (MSA) fault}
For each lines, there are three possible cases: no-error, stuck-at-0 or stuck-at-1. With K lines, there would be $\text{3}^\text{K}$ possible circuits, including $\text{3}^\text{K}$-1 faulty ones and 1 fault-free one.

    \end{solution}

    \part (5\%) Please show the similarity and differences of (single, multiple) fault numbers between the state transition fault model and the stuck-at fault model.
    \begin{solution}
    \paragraph{Single}
Both the numbers of faults of SST and SSA fault are multiples of the numbers of transitions and lines, respectively. However, the multiplier of the SST fault depends on the number of states, while the one of the SSA is fixed to 2 (sa0, sa1).
\paragraph{Multiple}
In the power of both the numbers of faults of SST and SSA fault. The exponents are the numbers of transitions and lines, respectively. However, the base of the SST fault depends on the number of states, while the one of the SSA is fixed to 3 (no-error, sa0, sa1).

    \end{solution}

  \end{parts}

  \question (20\%) Prove that for combinational circuits \textbf{faults dominance is a transitive relation}, i.e. if f dominates g and g dominates h, then f dominates h.
  \begin{solution}
  % The test pattern can be given as the table below. Each row is a clock tick. Assume $\text{Q}_\text{n}$ = 0 at the initial tick. {\color{red} (Todo: How to make such assumption?)} For all the following ticks, $\text{Q}_\text{n}$ is the $\text{Q}_\text{n+1}$ from the previous tick.
\begin{center}
  \begin{tabular}{c | c | c}
    T & $\text{Q}_\text{n+1}$ & Functions \\
    \hline
    0 & 0 & hold 0 \\
    1 & 1 & toggle to 1 \\
    0 & 1 & hold 1 \\
    1 & 0 & toggle to 0 \\
  \end{tabular}
\end{center}
If output does not meet the expectation, a functional fault is detected in the F/F. One of the hold or toggle to 0/1 functions could not behave properly.

  \end{solution}

  \question (55\%) In the circuit shown in Figure 1,
  The test pattern can be given as the table below. Each row is a clock tick, and at the initial tick, $\text{Q}_\text{n}$ is in the don't-care condition. For all the following ticks, $\text{Q}_\text{n}$ is the $\text{Q}_\text{n+1}$ from the previous tick.
\begin{center}
  \begin{tabular}{c c | c | c}
    J & K & $\text{Q}_\text{n+1}$ & Functions \\
    \hline
    0 & 1 & 0 & set 0 \\
    0 & 0 & 0 & hold 0 \\
    1 & 0 & 1 & set 1 \\
    0 & 0 & 1 & hold 1 \\
    1 & 1 & 0 & toggle to 0 \\
    1 & 1 & 1 & toggle to 1 \\
  \end{tabular}
\end{center}
If output does not meet the expectation, a functional fault is detected in the F/F. One of the set, hold or toggle to 0/1 functions could not behave properly.

  \begin{parts}
    \part (5\%) How many single stuck-at faults needed to be considered initially?
    \begin{solution}
    % In the figure below, the red circles indicate the missing crosspoint faults, and the upward arrows indicate the s-a-1 faults. The missing crosspoint in the AND plane causes growth fault, i.e., the lines having crosspoint with C will no longer depend on C.
\begin{center}
  \begin{minipage}{0.4\linewidth}
    \raggedleft
    \begin{circuitikz}[line width=.7pt]
      \ctikzset{logic ports=ieee}
    
      % A & A negate
\draw (0,0) node[label=A] (A) {} -- ++(0,-4.5);
\draw (0.5,-1) node[not port, rotate=-90, scale=0.4, circuitikz/ieeestd ports/not radius=.25] (A_neg) {};
\draw (A_neg.in) |- ++(-0.5,0.25) to[short, *-] ++(0,0);
\draw (A_neg.out) -- (0.5,-4.5);

% B & B negate
\draw (1,0) node[label=B] (B) {} -- ++(0,-4.5);
\draw (1.5,-1) node[not port, rotate=-90, scale=0.4, circuitikz/ieeestd ports/not radius=.25] (B_neg) {};
\draw (B_neg.in) |- ++(-0.5,0.25) to[short, *-] ++(0,0);
\draw (B_neg.out) -- (1.5,-4.5);

% C & C negate
\draw (2,0) node[label=C] (C) {} -- ++(0,-4.5);
\draw (2.5,-1) node[not port, rotate=-90, scale=0.4, circuitikz/ieeestd ports/not radius=.25] (C_neg) {};
\draw (C_neg.in) |- ++(-0.5,0.25) to[short, *-] ++(0,0);
\draw (C_neg.out) -- (2.5,-4.5);

% f1 & f2
\draw (4.0,0) node[label=f1] (f1) {} -- ++(0,-4.5);
\draw (4.5,0) node[label=f2] (f2) {} -- ++(0,-4.5);

% horizontal lines
\draw (-0.5,-1.5) -- ++(5.5,0);
\draw (-0.5,-2.0) -- ++(5.5,0);
\draw (-0.5,-2.5) -- ++(5.5,0);
\draw (-0.5,-3.0) -- ++(5.5,0);
\draw (-0.5,-3.5) -- ++(5.5,0);
\draw (-0.5,-4.0) -- ++(5.5,0);

% and plane
% AB
\draw (0,-1.5) to[short, *-*] ++(1,0);
% BC
\draw (1,-2.0) to[short, *-*] ++(1,0);
% CA
\draw (0,-2.5) to[short, *-*] ++(2,0);
% A_negB_neg
\draw (0.5,-3.0) to[short, *-*] ++(1,0);
% B_negC_neg
\draw (1.5,-3.5) to[short, *-*] ++(1,0);
% C_negA_neg
\draw (0.5,-4.0) to[short, *-*] ++(2,0);

    
      % f1 or plane
      \draw (4,-1.5) node[label=center:$\times$] {};
      \draw (4,-2.0) node[label=center:$\times$] {};
      \draw (4,-2.5) node[label=center:$\times$] {};
      \draw (4,-3.0) node[label=center:$\times$] {};
      \draw (4,-3.5) node[label=center:$\times$] {};
      \draw (4,-4.0) node[label=center:$\times$] {};
    
      % f2 or plane
      \draw (4.5,-1.5) node[label=center:$\times$] {};
      \draw (4.5,-2.0) node[label=center:$\times$] {};
      \draw (4.5,-2.5) node[label=center:$\times$] {};
    
      % fault
      \draw[color=red] (2,-1.5) circle[radius=6pt] {};
      \draw[color=red] (2,-2.0) circle[radius=6pt] {};
      \draw[color=red] (2,-2.5) circle[radius=6pt] {};
      \draw[color=red] (2,-3.0) circle[radius=6pt] {};
      \draw[color=red] (2,-3.5) circle[radius=6pt] {};
      \draw[color=red] (2,-4.0) circle[radius=6pt] {};
      
    \end{circuitikz}
  \end{minipage}
  \hfill
  \begin{minipage}{0.5\linewidth}
    \begin{circuitikz}[line width=.7pt]
      \ctikzset{logic ports=ieee}

      % A & A negate
\draw (0,0) node[label=A] (A) {} -- ++(0,-4.5);
\draw (0.5,-1) node[not port, rotate=-90, scale=0.4, circuitikz/ieeestd ports/not radius=.25] (A_neg) {};
\draw (A_neg.in) |- ++(-0.5,0.25) to[short, *-] ++(0,0);
\draw (A_neg.out) -- (0.5,-4.5);

% B & B negate
\draw (1,0) node[label=B] (B) {} -- ++(0,-4.5);
\draw (1.5,-1) node[not port, rotate=-90, scale=0.4, circuitikz/ieeestd ports/not radius=.25] (B_neg) {};
\draw (B_neg.in) |- ++(-0.5,0.25) to[short, *-] ++(0,0);
\draw (B_neg.out) -- (1.5,-4.5);

% C & C negate
\draw (2,0) node[label=C] (C) {} -- ++(0,-4.5);
\draw (2.5,-1) node[not port, rotate=-90, scale=0.4, circuitikz/ieeestd ports/not radius=.25] (C_neg) {};
\draw (C_neg.in) |- ++(-0.5,0.25) to[short, *-] ++(0,0);
\draw (C_neg.out) -- (2.5,-4.5);

% horizontal lines
\draw (-0.5,-1.5) -- ++(3.25,0) -- ++(0,0) node[and port, anchor=in 1, scale=0.35715] (p1) {};
\draw (-0.5,-1.7) -- ++(3.25,0) -- (p1.in 2);
\draw (-0.5,-2.0) -- ++(3.25,0) -- ++(0,0) node[and port, anchor=in 1, scale=0.35715] (p2) {};
\draw (-0.5,-2.2) -- ++(3.25,0);
\draw (-0.5,-2.5) -- ++(3.25,0) -- ++(0,0) node[and port, anchor=in 1, scale=0.35715] (p3) {};
\draw (-0.5,-2.7) -- ++(3.25,0);
\draw (-0.5,-3.0) -- ++(3.25,0) -- ++(0,0) node[and port, anchor=in 1, scale=0.35715] (p4) {};
\draw (-0.5,-3.2) -- ++(3.25,0);
\draw (-0.5,-3.5) -- ++(3.25,0) -- ++(0,0) node[and port, anchor=in 1, scale=0.35715] (p5) {};
\draw (-0.5,-3.7) -- ++(3.25,0);
\draw (-0.5,-4.0) -- ++(3.25,0) -- ++(0,0) node[and port, anchor=in 1, scale=0.35715] (p6) {};
\draw (-0.5,-4.2) -- ++(3.25,0);

% and plane
% AB
\draw (0,-1.5)  to[short, *-] ++(0,0);
\draw (1,-1.7) to[short, *-] ++(0,0);
% BC
\draw (1,-2.0)  to[short, *-] ++(0,0);
\draw (2,-2.2) to[short, *-] ++(0,0);
% CA
\draw (0,-2.5)  to[short, *-] ++(0,0);
\draw (2,-2.7) to[short, *-] ++(0,0);
% A_negB_neg
\draw (0.5,-3.0)  to[short, *-] ++(0,0);
\draw (1.5,-3.2) to[short, *-] ++(0,0);
% B_negC_neg
\draw (1.5,-3.5)  to[short, *-] ++(0,0);
\draw (2.5,-3.7) to[short, *-] ++(0,0);
% C_negA_neg
\draw (0.5,-4.0)  to[short, *-] ++(0,0);
\draw (2.5,-4.2) to[short, *-] ++(0,0);

    
      % f1
      \draw (5,-2.85) node[or port, scale=0.5, number inputs=6] (f1) {};
      \draw (p1.out) -- ++(0.2,0) |- (f1.in 1);
      \draw (p2.out) -- ++(0.1,0) |- (f1.in 2);
      \draw (p3.out) |- (f1.in 3);
      \draw (p4.out) |- (f1.in 4);
      \draw (p5.out) -- ++(0.1,0) |- (f1.in 5);
      \draw (p6.out) -- ++(0.2,0) |- (f1.in 6);
      \draw (f1.out) -- ++(0,0) node[label=right:f1] {};

      % f2
      \draw (5,-2) node[or port, scale=0.5, number inputs=3] (f2) {};

      \draw (f1.in 1) to[short,*-] (f2.in 3);

      \draw (f2.in 2) ++(-0.2,0) node (f2_in2_shifted) {} -- (f2.in 2);
      \draw (f1.in 2) ++(-0.2,0) to[short,*-] (f2_in2_shifted.center);

      \draw (f2.in 1) ++(-0.4,0) node (f2_in1_shifted) {} -- (f2.in 1);
      \draw (f1.in 3) ++(-0.4,0) to[short,*-] (f2_in1_shifted.center);

      \draw (f2.out) -- ++(0,0) node[label=right:f2] {};
    
      % fault
      \draw[color=red, -latexslim] (p2.in 2) ++(-0.05,-0.15) -- ++(0,0.3);
      \draw[color=red, -latexslim] (p3.in 2) ++(-0.05,-0.15) -- ++(0,0.3);
      \draw[color=red] (3.25,-1.5) node[label=s-a-1] {};
    \end{circuitikz}
  \end{minipage}
\end{center}

    \end{solution}

    \part (25\%) Applying the \textbf{check point theorem (incl. fault dominance)}, how many check point faults needed to be considered?
    \begin{solution}
    % In the figure below, the red circles indicate the missing crosspoint faults, and the upward arrows indicate the s-a-1 faults. The missing crosspoint in the AND plane causes growth fault, i.e., the lines having crosspoint with C will no longer depend on C.
\begin{center}
  \begin{minipage}{0.4\linewidth}
    \raggedleft
    \begin{circuitikz}[line width=.7pt]
      \ctikzset{logic ports=ieee}
    
      % A & A negate
\draw (0,0) node[label=A] (A) {} -- ++(0,-4.5);
\draw (0.5,-1) node[not port, rotate=-90, scale=0.4, circuitikz/ieeestd ports/not radius=.25] (A_neg) {};
\draw (A_neg.in) |- ++(-0.5,0.25) to[short, *-] ++(0,0);
\draw (A_neg.out) -- (0.5,-4.5);

% B & B negate
\draw (1,0) node[label=B] (B) {} -- ++(0,-4.5);
\draw (1.5,-1) node[not port, rotate=-90, scale=0.4, circuitikz/ieeestd ports/not radius=.25] (B_neg) {};
\draw (B_neg.in) |- ++(-0.5,0.25) to[short, *-] ++(0,0);
\draw (B_neg.out) -- (1.5,-4.5);

% C & C negate
\draw (2,0) node[label=C] (C) {} -- ++(0,-4.5);
\draw (2.5,-1) node[not port, rotate=-90, scale=0.4, circuitikz/ieeestd ports/not radius=.25] (C_neg) {};
\draw (C_neg.in) |- ++(-0.5,0.25) to[short, *-] ++(0,0);
\draw (C_neg.out) -- (2.5,-4.5);

% f1 & f2
\draw (4.0,0) node[label=f1] (f1) {} -- ++(0,-4.5);
\draw (4.5,0) node[label=f2] (f2) {} -- ++(0,-4.5);

% horizontal lines
\draw (-0.5,-1.5) -- ++(5.5,0);
\draw (-0.5,-2.0) -- ++(5.5,0);
\draw (-0.5,-2.5) -- ++(5.5,0);
\draw (-0.5,-3.0) -- ++(5.5,0);
\draw (-0.5,-3.5) -- ++(5.5,0);
\draw (-0.5,-4.0) -- ++(5.5,0);

% and plane
% AB
\draw (0,-1.5) to[short, *-*] ++(1,0);
% BC
\draw (1,-2.0) to[short, *-*] ++(1,0);
% CA
\draw (0,-2.5) to[short, *-*] ++(2,0);
% A_negB_neg
\draw (0.5,-3.0) to[short, *-*] ++(1,0);
% B_negC_neg
\draw (1.5,-3.5) to[short, *-*] ++(1,0);
% C_negA_neg
\draw (0.5,-4.0) to[short, *-*] ++(2,0);

    
      % f1 or plane
      \draw (4,-1.5) node[label=center:$\times$] {};
      \draw (4,-2.0) node[label=center:$\times$] {};
      \draw (4,-2.5) node[label=center:$\times$] {};
      \draw (4,-3.0) node[label=center:$\times$] {};
      \draw (4,-3.5) node[label=center:$\times$] {};
      \draw (4,-4.0) node[label=center:$\times$] {};
    
      % f2 or plane
      \draw (4.5,-1.5) node[label=center:$\times$] {};
      \draw (4.5,-2.0) node[label=center:$\times$] {};
      \draw (4.5,-2.5) node[label=center:$\times$] {};
    
      % fault
      \draw[color=red] (2,-1.5) circle[radius=6pt] {};
      \draw[color=red] (2,-2.0) circle[radius=6pt] {};
      \draw[color=red] (2,-2.5) circle[radius=6pt] {};
      \draw[color=red] (2,-3.0) circle[radius=6pt] {};
      \draw[color=red] (2,-3.5) circle[radius=6pt] {};
      \draw[color=red] (2,-4.0) circle[radius=6pt] {};
      
    \end{circuitikz}
  \end{minipage}
  \hfill
  \begin{minipage}{0.5\linewidth}
    \begin{circuitikz}[line width=.7pt]
      \ctikzset{logic ports=ieee}

      % A & A negate
\draw (0,0) node[label=A] (A) {} -- ++(0,-4.5);
\draw (0.5,-1) node[not port, rotate=-90, scale=0.4, circuitikz/ieeestd ports/not radius=.25] (A_neg) {};
\draw (A_neg.in) |- ++(-0.5,0.25) to[short, *-] ++(0,0);
\draw (A_neg.out) -- (0.5,-4.5);

% B & B negate
\draw (1,0) node[label=B] (B) {} -- ++(0,-4.5);
\draw (1.5,-1) node[not port, rotate=-90, scale=0.4, circuitikz/ieeestd ports/not radius=.25] (B_neg) {};
\draw (B_neg.in) |- ++(-0.5,0.25) to[short, *-] ++(0,0);
\draw (B_neg.out) -- (1.5,-4.5);

% C & C negate
\draw (2,0) node[label=C] (C) {} -- ++(0,-4.5);
\draw (2.5,-1) node[not port, rotate=-90, scale=0.4, circuitikz/ieeestd ports/not radius=.25] (C_neg) {};
\draw (C_neg.in) |- ++(-0.5,0.25) to[short, *-] ++(0,0);
\draw (C_neg.out) -- (2.5,-4.5);

% horizontal lines
\draw (-0.5,-1.5) -- ++(3.25,0) -- ++(0,0) node[and port, anchor=in 1, scale=0.35715] (p1) {};
\draw (-0.5,-1.7) -- ++(3.25,0) -- (p1.in 2);
\draw (-0.5,-2.0) -- ++(3.25,0) -- ++(0,0) node[and port, anchor=in 1, scale=0.35715] (p2) {};
\draw (-0.5,-2.2) -- ++(3.25,0);
\draw (-0.5,-2.5) -- ++(3.25,0) -- ++(0,0) node[and port, anchor=in 1, scale=0.35715] (p3) {};
\draw (-0.5,-2.7) -- ++(3.25,0);
\draw (-0.5,-3.0) -- ++(3.25,0) -- ++(0,0) node[and port, anchor=in 1, scale=0.35715] (p4) {};
\draw (-0.5,-3.2) -- ++(3.25,0);
\draw (-0.5,-3.5) -- ++(3.25,0) -- ++(0,0) node[and port, anchor=in 1, scale=0.35715] (p5) {};
\draw (-0.5,-3.7) -- ++(3.25,0);
\draw (-0.5,-4.0) -- ++(3.25,0) -- ++(0,0) node[and port, anchor=in 1, scale=0.35715] (p6) {};
\draw (-0.5,-4.2) -- ++(3.25,0);

% and plane
% AB
\draw (0,-1.5)  to[short, *-] ++(0,0);
\draw (1,-1.7) to[short, *-] ++(0,0);
% BC
\draw (1,-2.0)  to[short, *-] ++(0,0);
\draw (2,-2.2) to[short, *-] ++(0,0);
% CA
\draw (0,-2.5)  to[short, *-] ++(0,0);
\draw (2,-2.7) to[short, *-] ++(0,0);
% A_negB_neg
\draw (0.5,-3.0)  to[short, *-] ++(0,0);
\draw (1.5,-3.2) to[short, *-] ++(0,0);
% B_negC_neg
\draw (1.5,-3.5)  to[short, *-] ++(0,0);
\draw (2.5,-3.7) to[short, *-] ++(0,0);
% C_negA_neg
\draw (0.5,-4.0)  to[short, *-] ++(0,0);
\draw (2.5,-4.2) to[short, *-] ++(0,0);

    
      % f1
      \draw (5,-2.85) node[or port, scale=0.5, number inputs=6] (f1) {};
      \draw (p1.out) -- ++(0.2,0) |- (f1.in 1);
      \draw (p2.out) -- ++(0.1,0) |- (f1.in 2);
      \draw (p3.out) |- (f1.in 3);
      \draw (p4.out) |- (f1.in 4);
      \draw (p5.out) -- ++(0.1,0) |- (f1.in 5);
      \draw (p6.out) -- ++(0.2,0) |- (f1.in 6);
      \draw (f1.out) -- ++(0,0) node[label=right:f1] {};

      % f2
      \draw (5,-2) node[or port, scale=0.5, number inputs=3] (f2) {};

      \draw (f1.in 1) to[short,*-] (f2.in 3);

      \draw (f2.in 2) ++(-0.2,0) node (f2_in2_shifted) {} -- (f2.in 2);
      \draw (f1.in 2) ++(-0.2,0) to[short,*-] (f2_in2_shifted.center);

      \draw (f2.in 1) ++(-0.4,0) node (f2_in1_shifted) {} -- (f2.in 1);
      \draw (f1.in 3) ++(-0.4,0) to[short,*-] (f2_in1_shifted.center);

      \draw (f2.out) -- ++(0,0) node[label=right:f2] {};
    
      % fault
      \draw[color=red, -latexslim] (p2.in 2) ++(-0.05,-0.15) -- ++(0,0.3);
      \draw[color=red, -latexslim] (p3.in 2) ++(-0.05,-0.15) -- ++(0,0.3);
      \draw[color=red] (3.25,-1.5) node[label=s-a-1] {};
    \end{circuitikz}
  \end{minipage}
\end{center}

    \end{solution}

    \part (25\%) Using \textbf{fault dominance} and \textbf{fault equivalence} relations to further reduce the number of stuck-at faults? How many remaining faults needed to be considered?
    \begin{solution}
    % In the figure below, the red circles indicate the missing crosspoint faults, and the upward arrows indicate the s-a-1 faults. The missing crosspoint in the AND plane causes growth fault, i.e., the lines having crosspoint with C will no longer depend on C.
\begin{center}
  \begin{minipage}{0.4\linewidth}
    \raggedleft
    \begin{circuitikz}[line width=.7pt]
      \ctikzset{logic ports=ieee}
    
      % A & A negate
\draw (0,0) node[label=A] (A) {} -- ++(0,-4.5);
\draw (0.5,-1) node[not port, rotate=-90, scale=0.4, circuitikz/ieeestd ports/not radius=.25] (A_neg) {};
\draw (A_neg.in) |- ++(-0.5,0.25) to[short, *-] ++(0,0);
\draw (A_neg.out) -- (0.5,-4.5);

% B & B negate
\draw (1,0) node[label=B] (B) {} -- ++(0,-4.5);
\draw (1.5,-1) node[not port, rotate=-90, scale=0.4, circuitikz/ieeestd ports/not radius=.25] (B_neg) {};
\draw (B_neg.in) |- ++(-0.5,0.25) to[short, *-] ++(0,0);
\draw (B_neg.out) -- (1.5,-4.5);

% C & C negate
\draw (2,0) node[label=C] (C) {} -- ++(0,-4.5);
\draw (2.5,-1) node[not port, rotate=-90, scale=0.4, circuitikz/ieeestd ports/not radius=.25] (C_neg) {};
\draw (C_neg.in) |- ++(-0.5,0.25) to[short, *-] ++(0,0);
\draw (C_neg.out) -- (2.5,-4.5);

% f1 & f2
\draw (4.0,0) node[label=f1] (f1) {} -- ++(0,-4.5);
\draw (4.5,0) node[label=f2] (f2) {} -- ++(0,-4.5);

% horizontal lines
\draw (-0.5,-1.5) -- ++(5.5,0);
\draw (-0.5,-2.0) -- ++(5.5,0);
\draw (-0.5,-2.5) -- ++(5.5,0);
\draw (-0.5,-3.0) -- ++(5.5,0);
\draw (-0.5,-3.5) -- ++(5.5,0);
\draw (-0.5,-4.0) -- ++(5.5,0);

% and plane
% AB
\draw (0,-1.5) to[short, *-*] ++(1,0);
% BC
\draw (1,-2.0) to[short, *-*] ++(1,0);
% CA
\draw (0,-2.5) to[short, *-*] ++(2,0);
% A_negB_neg
\draw (0.5,-3.0) to[short, *-*] ++(1,0);
% B_negC_neg
\draw (1.5,-3.5) to[short, *-*] ++(1,0);
% C_negA_neg
\draw (0.5,-4.0) to[short, *-*] ++(2,0);

    
      % f1 or plane
      \draw (4,-1.5) node[label=center:$\times$] {};
      \draw (4,-2.0) node[label=center:$\times$] {};
      \draw (4,-2.5) node[label=center:$\times$] {};
      \draw (4,-3.0) node[label=center:$\times$] {};
      \draw (4,-3.5) node[label=center:$\times$] {};
      \draw (4,-4.0) node[label=center:$\times$] {};
    
      % f2 or plane
      \draw (4.5,-1.5) node[label=center:$\times$] {};
      \draw (4.5,-2.0) node[label=center:$\times$] {};
      \draw (4.5,-2.5) node[label=center:$\times$] {};
    
      % fault
      \draw[color=red] (2,-1.5) circle[radius=6pt] {};
      \draw[color=red] (2,-2.0) circle[radius=6pt] {};
      \draw[color=red] (2,-2.5) circle[radius=6pt] {};
      \draw[color=red] (2,-3.0) circle[radius=6pt] {};
      \draw[color=red] (2,-3.5) circle[radius=6pt] {};
      \draw[color=red] (2,-4.0) circle[radius=6pt] {};
      
    \end{circuitikz}
  \end{minipage}
  \hfill
  \begin{minipage}{0.5\linewidth}
    \begin{circuitikz}[line width=.7pt]
      \ctikzset{logic ports=ieee}

      % A & A negate
\draw (0,0) node[label=A] (A) {} -- ++(0,-4.5);
\draw (0.5,-1) node[not port, rotate=-90, scale=0.4, circuitikz/ieeestd ports/not radius=.25] (A_neg) {};
\draw (A_neg.in) |- ++(-0.5,0.25) to[short, *-] ++(0,0);
\draw (A_neg.out) -- (0.5,-4.5);

% B & B negate
\draw (1,0) node[label=B] (B) {} -- ++(0,-4.5);
\draw (1.5,-1) node[not port, rotate=-90, scale=0.4, circuitikz/ieeestd ports/not radius=.25] (B_neg) {};
\draw (B_neg.in) |- ++(-0.5,0.25) to[short, *-] ++(0,0);
\draw (B_neg.out) -- (1.5,-4.5);

% C & C negate
\draw (2,0) node[label=C] (C) {} -- ++(0,-4.5);
\draw (2.5,-1) node[not port, rotate=-90, scale=0.4, circuitikz/ieeestd ports/not radius=.25] (C_neg) {};
\draw (C_neg.in) |- ++(-0.5,0.25) to[short, *-] ++(0,0);
\draw (C_neg.out) -- (2.5,-4.5);

% horizontal lines
\draw (-0.5,-1.5) -- ++(3.25,0) -- ++(0,0) node[and port, anchor=in 1, scale=0.35715] (p1) {};
\draw (-0.5,-1.7) -- ++(3.25,0) -- (p1.in 2);
\draw (-0.5,-2.0) -- ++(3.25,0) -- ++(0,0) node[and port, anchor=in 1, scale=0.35715] (p2) {};
\draw (-0.5,-2.2) -- ++(3.25,0);
\draw (-0.5,-2.5) -- ++(3.25,0) -- ++(0,0) node[and port, anchor=in 1, scale=0.35715] (p3) {};
\draw (-0.5,-2.7) -- ++(3.25,0);
\draw (-0.5,-3.0) -- ++(3.25,0) -- ++(0,0) node[and port, anchor=in 1, scale=0.35715] (p4) {};
\draw (-0.5,-3.2) -- ++(3.25,0);
\draw (-0.5,-3.5) -- ++(3.25,0) -- ++(0,0) node[and port, anchor=in 1, scale=0.35715] (p5) {};
\draw (-0.5,-3.7) -- ++(3.25,0);
\draw (-0.5,-4.0) -- ++(3.25,0) -- ++(0,0) node[and port, anchor=in 1, scale=0.35715] (p6) {};
\draw (-0.5,-4.2) -- ++(3.25,0);

% and plane
% AB
\draw (0,-1.5)  to[short, *-] ++(0,0);
\draw (1,-1.7) to[short, *-] ++(0,0);
% BC
\draw (1,-2.0)  to[short, *-] ++(0,0);
\draw (2,-2.2) to[short, *-] ++(0,0);
% CA
\draw (0,-2.5)  to[short, *-] ++(0,0);
\draw (2,-2.7) to[short, *-] ++(0,0);
% A_negB_neg
\draw (0.5,-3.0)  to[short, *-] ++(0,0);
\draw (1.5,-3.2) to[short, *-] ++(0,0);
% B_negC_neg
\draw (1.5,-3.5)  to[short, *-] ++(0,0);
\draw (2.5,-3.7) to[short, *-] ++(0,0);
% C_negA_neg
\draw (0.5,-4.0)  to[short, *-] ++(0,0);
\draw (2.5,-4.2) to[short, *-] ++(0,0);

    
      % f1
      \draw (5,-2.85) node[or port, scale=0.5, number inputs=6] (f1) {};
      \draw (p1.out) -- ++(0.2,0) |- (f1.in 1);
      \draw (p2.out) -- ++(0.1,0) |- (f1.in 2);
      \draw (p3.out) |- (f1.in 3);
      \draw (p4.out) |- (f1.in 4);
      \draw (p5.out) -- ++(0.1,0) |- (f1.in 5);
      \draw (p6.out) -- ++(0.2,0) |- (f1.in 6);
      \draw (f1.out) -- ++(0,0) node[label=right:f1] {};

      % f2
      \draw (5,-2) node[or port, scale=0.5, number inputs=3] (f2) {};

      \draw (f1.in 1) to[short,*-] (f2.in 3);

      \draw (f2.in 2) ++(-0.2,0) node (f2_in2_shifted) {} -- (f2.in 2);
      \draw (f1.in 2) ++(-0.2,0) to[short,*-] (f2_in2_shifted.center);

      \draw (f2.in 1) ++(-0.4,0) node (f2_in1_shifted) {} -- (f2.in 1);
      \draw (f1.in 3) ++(-0.4,0) to[short,*-] (f2_in1_shifted.center);

      \draw (f2.out) -- ++(0,0) node[label=right:f2] {};
    
      % fault
      \draw[color=red, -latexslim] (p2.in 2) ++(-0.05,-0.15) -- ++(0,0.3);
      \draw[color=red, -latexslim] (p3.in 2) ++(-0.05,-0.15) -- ++(0,0.3);
      \draw[color=red] (3.25,-1.5) node[label=s-a-1] {};
    \end{circuitikz}
  \end{minipage}
\end{center}

    \end{solution}

  \end{parts}
\end{questions}
\end{document}

