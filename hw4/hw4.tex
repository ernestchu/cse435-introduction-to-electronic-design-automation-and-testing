\documentclass[12pt,answers]{exam}
\usepackage{fontspec,graphicx,circuitikz,amsmath,caption,xcolor,ctable}
\setmainfont{Times New Roman}

\headheight 8pt \headsep 20pt \footskip 30pt
\textheight 9in \textwidth 6.5in
\oddsidemargin 0in \evensidemargin 0in
\topmargin -.35in

\begin{document}
\begin{center}
\LARGE CSE435 Introduction to EDA \& Testing - Spring 2022 \\
\Large Homework Assignment \#4 \\
\Large Shao-Hsuan Chu - B073040018 \\
\end{center}
\bigskip

\begin{questions}
  \question (25\%)
  \begin{parts}
    \part (10\%) For state transition fault model, explain why there are M(N-1) faults for a M-transition N-state machine. Similarly explain why there are $\text{N}^\text{M}$-1 multiple state transition faults.
    \begin{solution}
    Denote M and N as the number of transitions and states, respectively. 
\paragraph{Single-state-transition (SST) fault}
For each transition, the possible destination could be either one of the states, so there are (N-1) faulty cases and 1 faultless case. The SST could occur at either one of the transitions, producing M(N-1) distinguishable faults.
\paragraph{Multiple-state-transition (MST) fault}
For each transition, the possible destination could be either one of the states, so there are N cases. Since each transition is independent to other transitions, there are $\text{N}^\text{M}$ distinguishable state-transition diagrams. Disregarding 1 faultless diagram, the number of faults in MST is thus $\text{N}^\text{M}$-1.

    \end{solution}

    \part (10\%) For stuck-at fault model, explain why there are 2K single stuck-at faults. Similarly explain why there are $\text{3}^\text{K}$-1 multiple stuck-at faults.
    \begin{solution}
    Denote K as the number of lines in the circuit.

\paragraph{Single-stuck-at (SSA) fault}
The SSA fault would occur at either one of the lines, and in the SSA faults, there will be the case of stuck-at-0 and stuck-at-1. So the number of distinguishable SSA faults would be 2K.
\paragraph{Multiple-stuck-at (MSA) fault}
For each lines, there are three possible cases: no-error, stuck-at-0 or stuck-at-1. With K lines, there would be $\text{3}^\text{K}$ possible circuits, including $\text{3}^\text{K}$-1 faulty ones and 1 fault-free one.

    \end{solution}

    \part (5\%) Please show the similarity and differences of (single, multiple) fault numbers between the state transition fault model and the stuck-at fault model.
    \begin{solution}
    \paragraph{Single}
Both the numbers of faults of SST and SSA fault are multiples of the numbers of transitions and lines, respectively. However, the multiplier of the SST fault depends on the number of states, while the one of the SSA is fixed to 2 (sa0, sa1).
\paragraph{Multiple}
In the power of both the numbers of faults of SST and SSA fault. The exponents are the numbers of transitions and lines, respectively. However, the base of the SST fault depends on the number of states, while the one of the SSA is fixed to 3 (no-error, sa0, sa1).

    \end{solution}

  \end{parts}

  \question (20\%) Prove that for combinational circuits \textbf{faults dominance is a transitive relation}, i.e. if f dominates g and g dominates h, then f dominates h.
  \begin{solution}
  The test pattern can be given as the table below. Each row is a clock tick. Assume $\text{Q}_\text{n}$ = 0 at the initial tick. {\color{red} (Todo: How to make such assumption?)} For all the following ticks, $\text{Q}_\text{n}$ is the $\text{Q}_\text{n+1}$ from the previous tick.
\begin{center}
  \begin{tabular}{c | c | c}
    T & $\text{Q}_\text{n+1}$ & Functions \\
    \hline
    0 & 0 & hold 0 \\
    1 & 1 & toggle to 1 \\
    0 & 1 & hold 1 \\
    1 & 0 & toggle to 0 \\
  \end{tabular}
\end{center}
If output does not meet the expectation, a functional fault is detected in the F/F. One of the hold or toggle to 0/1 functions could not behave properly.

  \end{solution}

  \question (55\%) In the circuit shown in Figure 1,
  The test pattern can be given as the table below. Each row is a clock tick, and at the initial tick, $\text{Q}_\text{n}$ is in the don't-care condition. For all the following ticks, $\text{Q}_\text{n}$ is the $\text{Q}_\text{n+1}$ from the previous tick.
\begin{center}
  \begin{tabular}{c c | c | c}
    J & K & $\text{Q}_\text{n+1}$ & Functions \\
    \hline
    0 & 1 & 0 & set 0 \\
    0 & 0 & 0 & hold 0 \\
    1 & 0 & 1 & set 1 \\
    0 & 0 & 1 & hold 1 \\
    1 & 1 & 0 & toggle to 0 \\
    1 & 1 & 1 & toggle to 1 \\
  \end{tabular}
\end{center}
If output does not meet the expectation, a functional fault is detected in the F/F. One of the set, hold or toggle to 0/1 functions could not behave properly.

  \begin{parts}
    \part (5\%) How many single stuck-at faults needed to be considered initially?
    \begin{solution}
    There are 12 lines in the circuit. Therefore, there are 2 $\times$ 12 = 24 SSA faults initially.

    \end{solution}

    \part (25\%) Applying the \textbf{check point theorem (incl. fault dominance)}, how many check point faults needed to be considered?
    \begin{solution}
    Primary inputs and fanout branches form a sufficient set of checkpoints in an irredundant combinational circuit. Therefore, there are 7 checkpoints in this circuit, as shown in Figure 2.
\begin{center}
\begin{circuitikz}[line width=.7pt]
  \draw (0,0) node[and port] (and1) {};
\draw (and1.in 1) -- ++(-0.5,0) node[label=left:a] {};
\draw (and1.in 2) -- ++(-0.5,0) node[label=left:b] {};

\draw (and1.out) -- ++(1,0) node[nor port, anchor=in 1] (nor1) {};
\draw (nor1) ++(0,-2) node[and port] (and2) {};

\draw (and1.in 1) ++(-0.5,-1.5) node[label=left:c] (c) {};
\draw (c) -| (nor1.in 2);
\draw (c) -| (and2.in 1);

\draw (c |- and2.in 2) node[label=left:d] {} -- (and2.in 2);

\draw (nor1.out) -| ++(2,-1) node[or port, anchor=in 1] (or2) {};
\draw (and2.out) -| ++(0.3,-0.5) node[or port, anchor=in 1] (or1) {};
\draw (or1.in 2) -- ++(0,0) node[label=left:e] {};
\draw (or1.out) -| (or2.in 2);

\draw (or2.out) -- ++(0,0) node[label=right:m] {};

\draw (nor1.in 1) node[label=f] {};
\draw (nor1.in 2) node[label=left:g] {};
\draw (and2.in 1) node[label=left:h] {};
\draw (or1.in 1) node[label=left:i] {};
\draw (or2.in 1) node[label=left:j] {};
\draw (or2.in 2) node[label=left:k] {};

  \draw[color=red] (and1.in 1) ++(0,0) to[short, *-] ++(0,0);
  \draw[color=red] (and1.in 2) ++(0,0) to[short, *-] ++(0,0);
  \draw[color=red] (nor1.in 2) ++(0,0) to[short, *-] ++(0,0);
  \draw[color=red] (and2.in 1) ++(0,0) to[short, *-] ++(0,0);
  \draw[color=red] (and2.in 2) ++(0,0) to[short, *-] ++(0,0);
  \draw[color=red] (or1.in 2) ++(0,0) to[short, *-] ++(0,0);
  \draw[color=red] (and1.in 1) ++(0,-1.5) to[short, *-] ++(0,0);
\end{circuitikz}
\captionof{figure}{}
\end{center}

These 7 checkpoints could produce 2 $\times$ 7 = 14 SSA faults.

\begin{center}
  \begin{tabular}{lll}
    \specialrule{.1em}{.05em}{.05em} 
    Gate & Equivalence Set(s) &	Dominant Set(s) \\
    \hline
    OR & \{a-sa1, b-sa1, out-sa1\} & \{out-sa0: a-sa0, b-sa0\} \\
    NOR & \{a-sa1, b-sa1, out-sa0\} & \{out-sa1: a-sa0, b-sa0\} \\
    AND & \{a-sa0, b-sa0, out-sa0\} & \{out-sa1: a-sa1, b-sa1\} \\
    NAND & \{a-sa0, b-sa0, out-sa1\} & \{out-sa0: a-sa1, b-sa1\} \\
    Buffer & \vtop{\hbox{\strut \{in-sa0, out-sa0\}}\hbox{\strut \{in-sa1, out-sa1\}}} & null \\
    NOT & \vtop{\hbox{\strut \{in-sa1, out-sa0\}}\hbox{\strut \{in-sa0, out-sa1\}}} & null \\
    \specialrule{.1em}{.05em}{.05em} 
  \end{tabular}
\end{center}

% With fault dominance, s-a-1 at \textbf{a} and s-a-1 at \textbf{b} are dominated by s-a-1 at \textbf{f}, and we can replace the former two with the latter. Likewise, s-a-1 at \textbf{h} and s-a-1 at \textbf{d} are dominated by s-a-1 at \textbf{i}. As a result, the number of faults is reduced by 2, leaving 10 SSA faults after applying checkpoint theorem and fault dominance.


    \end{solution}

    \part (25\%) Using \textbf{fault dominance} and \textbf{fault equivalence} relations to further reduce the number of stuck-at faults? How many remaining faults needed to be considered?
    \begin{solution}
    % From Figure 9 to Figure 14 below, the red arrows indicate the necessary stuck-at faults, while the black ones indicates the collapsed faults after each steps.

\begin{center}
\begin{circuitikz}[line width=.7pt]
  \draw (0,0) node[and port] (and1) {};
\draw (and1.in 1) -- ++(-0.5,0) node[label=left:a] {};
\draw (and1.in 2) -- ++(-0.5,0) node[label=left:b] {};

\draw (and1.out) -- ++(1,0) node[nor port, anchor=in 1] (nor1) {};
\draw (nor1) ++(0,-2) node[and port] (and2) {};

\draw (and1.in 1) ++(-0.5,-1.5) node[label=left:c] (c) {};
\draw (c) -| (nor1.in 2);
\draw (c) -| (and2.in 1);

\draw (c |- and2.in 2) node[label=left:d] {} -- (and2.in 2);

\draw (nor1.out) -| ++(2,-1) node[or port, anchor=in 1] (or2) {};
\draw (and2.out) -| ++(0.3,-0.5) node[or port, anchor=in 1] (or1) {};
\draw (or1.in 2) -- ++(0,0) node[label=left:e] {};
\draw (or1.out) -| (or2.in 2);

\draw (or2.out) -- ++(0,0) node[label=right:m] {};

\draw (nor1.in 1) node[label=f] {};
\draw (nor1.in 2) node[label=left:g] {};
\draw (and2.in 1) node[label=left:h] {};
\draw (or1.in 1) node[label=left:i] {};
\draw (or2.in 1) node[label=left:j] {};
\draw (or2.in 2) node[label=left:k] {};

  % G1 in 1
  \draw[color=red, -latexslim] (and1.in 1) ++(0,0.25) -- ++(0,-0.5);
  \draw[color=red, -latexslim] (and1.in 1) ++(0.1,-0.25) -- ++(0,0.5);
  % G1 in 2
  \draw[color=red, -latexslim] (and1.in 2) ++(0,0.25) -- ++(0,-0.5);
  \draw[color=red, -latexslim] (and1.in 2) ++(0.1,-0.25) -- ++(0,0.5);
  % c
  \draw[color=red, -latexslim] (and1.in 1) ++(0,-1.25) -- ++(0,-0.5);
  \draw[color=red, -latexslim] (and1.in 1) ++(0.1,-1.75) -- ++(0,0.5);
  % G2 in 1
  \draw[color=red, -latexslim] (nor1.in 1) ++(0,0.25) -- ++(0,-0.5);
  \draw[color=red, -latexslim] (nor1.in 1) ++(0.1,-0.25) -- ++(0,0.5);
  % G2 in 2
  \draw[color=red, -latexslim] (nor1.in 2) ++(0,0.25) -- ++(0,-0.5);
  \draw[color=red, -latexslim] (nor1.in 2) ++(0.1,-0.25) -- ++(0,0.5);
  % G3 in 1
  \draw[color=red, -latexslim] (and2.in 1) ++(0,0.25) -- ++(0,-0.5);
  \draw[color=red, -latexslim] (and2.in 1) ++(0.1,-0.25) -- ++(0,0.5);
  % G3 in 2
  \draw[color=red, -latexslim] (and2.in 2) ++(0,0.25) -- ++(0,-0.5);
  \draw[color=red, -latexslim] (and2.in 2) ++(0.1,-0.25) -- ++(0,0.5);
  % G4 in 1
  \draw[color=red, -latexslim] (or1.in 1) ++(0,0.25) -- ++(0,-0.5);
  \draw[color=red, -latexslim] (or1.in 1) ++(0.1,-0.25) -- ++(0,0.5);
  % G4 in 2
  \draw[color=red, -latexslim] (or1.in 2) ++(0,0.25) -- ++(0,-0.5);
  \draw[color=red, -latexslim] (or1.in 2) ++(0.1,-0.25) -- ++(0,0.5);
  % G5 in 1
  \draw[color=red, -latexslim] (or2.in 1) ++(0,0.25) -- ++(0,-0.5);
  \draw[color=red, -latexslim] (or2.in 1) ++(0.1,-0.25) -- ++(0,0.5);
  % G5 in 2
  \draw[color=red, -latexslim] (or2.in 2) ++(0,0.25) -- ++(0,-0.5);
  \draw[color=red, -latexslim] (or2.in 2) ++(0.1,-0.25) -- ++(0,0.5);
  % G5 out
  \draw[color=red, -latexslim] (or2.out) ++(0,0.25) -- ++(0,-0.5);
  \draw[color=red, -latexslim] (or2.out) ++(0.1,-0.25) -- ++(0,0.5);

  \draw (and1) node[left=5pt] {G1};
  \draw (nor1) node[left=5pt] {G2};
  \draw (and2) node[left=5pt] {G3};
  \draw (or1) node[left=5pt] {G4};
  \draw (or2) node[left=5pt] {G5};
\end{circuitikz}
\captionof{figure}{Initial 24 stuck-at faults.}
\end{center}

\begin{center}
\begin{circuitikz}[line width=.7pt]
  \draw (0,0) node[and port] (and1) {};
\draw (and1.in 1) -- ++(-0.5,0) node[label=left:a] {};
\draw (and1.in 2) -- ++(-0.5,0) node[label=left:b] {};

\draw (and1.out) -- ++(1,0) node[nor port, anchor=in 1] (nor1) {};
\draw (nor1) ++(0,-2) node[and port] (and2) {};

\draw (and1.in 1) ++(-0.5,-1.5) node[label=left:c] (c) {};
\draw (c) -| (nor1.in 2);
\draw (c) -| (and2.in 1);

\draw (c |- and2.in 2) node[label=left:d] {} -- (and2.in 2);

\draw (nor1.out) -| ++(2,-1) node[or port, anchor=in 1] (or2) {};
\draw (and2.out) -| ++(0.3,-0.5) node[or port, anchor=in 1] (or1) {};
\draw (or1.in 2) -- ++(0,0) node[label=left:e] {};
\draw (or1.out) -| (or2.in 2);

\draw (or2.out) -- ++(0,0) node[label=right:m] {};

\draw (nor1.in 1) node[label=f] {};
\draw (nor1.in 2) node[label=left:g] {};
\draw (and2.in 1) node[label=left:h] {};
\draw (or1.in 1) node[label=left:i] {};
\draw (or2.in 1) node[label=left:j] {};
\draw (or2.in 2) node[label=left:k] {};

  % G1 in 1
  \draw[color=red, -latexslim] (and1.in 1) ++(0,0.25) -- ++(0,-0.5);
  \draw[color=red, -latexslim] (and1.in 1) ++(0.1,-0.25) -- ++(0,0.5);
  % G1 in 2
  \draw[color=red, -latexslim] (and1.in 2) ++(0,0.25) -- ++(0,-0.5);
  \draw[color=red, -latexslim] (and1.in 2) ++(0.1,-0.25) -- ++(0,0.5);
  % c
  \draw[color=red, -latexslim] (and1.in 1) ++(0,-1.25) -- ++(0,-0.5);
  \draw[color=red, -latexslim] (and1.in 1) ++(0.1,-1.75) -- ++(0,0.5);
  % G2 in 1
  \draw[color=red, -latexslim] (nor1.in 1) ++(0,0.25) -- ++(0,-0.5);
  \draw[color=red, -latexslim] (nor1.in 1) ++(0.1,-0.25) -- ++(0,0.5);
  % G2 in 2
  \draw[color=red, -latexslim] (nor1.in 2) ++(0,0.25) -- ++(0,-0.5);
  \draw[color=red, -latexslim] (nor1.in 2) ++(0.1,-0.25) -- ++(0,0.5);
  % G3 in 1
  \draw[color=red, -latexslim] (and2.in 1) ++(0,0.25) -- ++(0,-0.5);
  \draw[color=red, -latexslim] (and2.in 1) ++(0.1,-0.25) -- ++(0,0.5);
  % G3 in 2
  \draw[color=red, -latexslim] (and2.in 2) ++(0,0.25) -- ++(0,-0.5);
  \draw[color=red, -latexslim] (and2.in 2) ++(0.1,-0.25) -- ++(0,0.5);
  % G4 in 1
  \draw[color=red, -latexslim] (or1.in 1) ++(0,0.25) -- ++(0,-0.5);
  \draw[color=red, -latexslim] (or1.in 1) ++(0.1,-0.25) -- ++(0,0.5);
  % G4 in 2
  \draw[color=red, -latexslim] (or1.in 2) ++(0,0.25) -- ++(0,-0.5);
  \draw[color=red, -latexslim] (or1.in 2) ++(0.1,-0.25) -- ++(0,0.5);
  % G5 in 1
  \draw[color=red, -latexslim] (or2.in 1) ++(0,0.25) -- ++(0,-0.5);
  \draw[color=red, -latexslim] (or2.in 1) ++(0.1,-0.25) -- ++(0,0.5);
  % G5 in 2
  \draw[color=red, -latexslim] (or2.in 2) ++(0,0.25) -- ++(0,-0.5);
  \draw[-latexslim] (or2.in 2) ++(0.1,-0.25) -- ++(0,0.5);
  % G5 out
  \draw[-latexslim] (or2.out) ++(0,0.25) -- ++(0,-0.5);
  \draw[-latexslim] (or2.out) ++(0.1,-0.25) -- ++(0,0.5);

  \draw (and1) node[left=5pt] {G1};
  \draw (nor1) node[left=5pt] {G2};
  \draw (and2) node[left=5pt] {G3};
  \draw (or1) node[left=5pt] {G4};
  \draw (or2) node[left=5pt] {G5};
\end{circuitikz}
\captionof{figure}{At G5, m-sa0 dominates j-sa0 and k-sa0. k-sa1 and m-sa1 are equivalent to j-sa1.}
\end{center}

\begin{center}
\begin{circuitikz}[line width=.7pt]
  \draw (0,0) node[and port] (and1) {};
\draw (and1.in 1) -- ++(-0.5,0) node[label=left:a] {};
\draw (and1.in 2) -- ++(-0.5,0) node[label=left:b] {};

\draw (and1.out) -- ++(1,0) node[nor port, anchor=in 1] (nor1) {};
\draw (nor1) ++(0,-2) node[and port] (and2) {};

\draw (and1.in 1) ++(-0.5,-1.5) node[label=left:c] (c) {};
\draw (c) -| (nor1.in 2);
\draw (c) -| (and2.in 1);

\draw (c |- and2.in 2) node[label=left:d] {} -- (and2.in 2);

\draw (nor1.out) -| ++(2,-1) node[or port, anchor=in 1] (or2) {};
\draw (and2.out) -| ++(0.3,-0.5) node[or port, anchor=in 1] (or1) {};
\draw (or1.in 2) -- ++(0,0) node[label=left:e] {};
\draw (or1.out) -| (or2.in 2);

\draw (or2.out) -- ++(0,0) node[label=right:m] {};

\draw (nor1.in 1) node[label=f] {};
\draw (nor1.in 2) node[label=left:g] {};
\draw (and2.in 1) node[label=left:h] {};
\draw (or1.in 1) node[label=left:i] {};
\draw (or2.in 1) node[label=left:j] {};
\draw (or2.in 2) node[label=left:k] {};

  \draw[color=red, -latexslim] (and1.in 1) ++(0,0.25) -- ++(0,-0.5);
  \draw[color=red, -latexslim] (and1.in 1) ++(0.1,-0.25) -- ++(0,0.5);

  \draw[color=red, -latexslim] (and1.in 2) ++(0,0.25) -- ++(0,-0.5);
  \draw[color=red, -latexslim] (and1.in 2) ++(0.1,-0.25) -- ++(0,0.5);

  \draw[color=red, -latexslim] (and1.in 1) ++(0,-1.25) -- ++(0,-0.5);
  \draw[color=red, -latexslim] (and1.in 1) ++(0.1,-1.75) -- ++(0,0.5);

  \draw[color=red, -latexslim] (nor1.in 2) ++(0,0.25) -- ++(0,-0.5);
  \draw[color=red, -latexslim] (nor1.in 2) ++(0.1,-0.25) -- ++(0,0.5);

  \draw[color=red, -latexslim] (and2.in 1) ++(0,0.25) -- ++(0,-0.5);
  \draw[color=red, -latexslim] (and2.in 1) ++(0.1,-0.25) -- ++(0,0.5);

  \draw[color=red, -latexslim] (and2.in 2) ++(0,0.25) -- ++(0,-0.5);
  \draw[color=red, -latexslim] (and2.in 2) ++(0.1,-0.25) -- ++(0,0.5);

  \draw[color=red, -latexslim] (or1.in 2) ++(0,0.25) -- ++(0,-0.5);
  \draw[-latexslim] (or1.in 2) ++(0.1,-0.25) -- ++(0,0.5);
  \draw[color=red, -latexslim] (or1.in 1) ++(0.1,-0.25) -- ++(0,0.5);

  \draw (and1) node[left=5pt] {G1};
  \draw (nor1) node[left=5pt] {G2};
  \draw (and2) node[left=5pt] {G3};
  \draw (or1) node[left=5pt] {G4};
  \draw (or2) node[left=5pt] {G5};
\end{circuitikz}
\captionof{figure}{At G4, e-sa1 is equivalent to i-sa1.}
\end{center}

\begin{center}
\begin{circuitikz}[line width=.7pt]
  \draw (0,0) node[and port] (and1) {};
\draw (and1.in 1) -- ++(-0.5,0) node[label=left:a] {};
\draw (and1.in 2) -- ++(-0.5,0) node[label=left:b] {};

\draw (and1.out) -- ++(1,0) node[nor port, anchor=in 1] (nor1) {};
\draw (nor1) ++(0,-2) node[and port] (and2) {};

\draw (and1.in 1) ++(-0.5,-1.5) node[label=left:c] (c) {};
\draw (c) -| (nor1.in 2);
\draw (c) -| (and2.in 1);

\draw (c |- and2.in 2) node[label=left:d] {} -- (and2.in 2);

\draw (nor1.out) -| ++(2,-1) node[or port, anchor=in 1] (or2) {};
\draw (and2.out) -| ++(0.3,-0.5) node[or port, anchor=in 1] (or1) {};
\draw (or1.in 2) -- ++(0,0) node[label=left:e] {};
\draw (or1.out) -| (or2.in 2);

\draw (or2.out) -- ++(0,0) node[label=right:m] {};

\draw (nor1.in 1) node[label=f] {};
\draw (nor1.in 2) node[label=left:g] {};
\draw (and2.in 1) node[label=left:h] {};
\draw (or1.in 1) node[label=left:i] {};
\draw (or2.in 1) node[label=left:j] {};
\draw (or2.in 2) node[label=left:k] {};

  \draw[color=red, -latexslim] (and1.in 1) ++(0,0.25) -- ++(0,-0.5);
  \draw[color=red, -latexslim] (and1.in 1) ++(0.1,-0.25) -- ++(0,0.5);

  \draw[color=red, -latexslim] (and1.in 2) ++(0,0.25) -- ++(0,-0.5);
  \draw[color=red, -latexslim] (and1.in 2) ++(0.1,-0.25) -- ++(0,0.5);

  \draw[color=red, -latexslim] (and1.in 1) ++(0,-1.25) -- ++(0,-0.5);
  \draw[color=red, -latexslim] (and1.in 1) ++(0.1,-1.75) -- ++(0,0.5);

  \draw[color=red, -latexslim] (nor1.in 2) ++(0,0.25) -- ++(0,-0.5);
  \draw[color=red, -latexslim] (nor1.in 2) ++(0.1,-0.25) -- ++(0,0.5);

  \draw[-latexslim] (and2.in 1) ++(0,0.25) -- ++(0,-0.5);
  \draw[color=red, -latexslim] (and2.in 1) ++(0.1,-0.25) -- ++(0,0.5);

  \draw[color=red, -latexslim] (and2.in 2) ++(0,0.25) -- ++(0,-0.5);
  \draw[color=red, -latexslim] (and2.in 2) ++(0.1,-0.25) -- ++(0,0.5);

  \draw[color=red, -latexslim] (or1.in 2) ++(0,0.25) -- ++(0,-0.5);
  \draw[-latexslim] (or1.in 2) ++(0.1,-0.25) -- ++(0,0.5);
  \draw[-latexslim] (or1.in 1) ++(0.1,-0.25) -- ++(0,0.5);

  \draw (and1) node[left=5pt] {G1};
  \draw (nor1) node[left=5pt] {G2};
  \draw (and2) node[left=5pt] {G3};
  \draw (or1) node[left=5pt] {G4};
  \draw (or2) node[left=5pt] {G5};
\end{circuitikz}
\captionof{figure}{At G3, i-sa1 dominates h-sa1 and d-sa1. h-sa0 is equivalent to d-sa0. }
\end{center}

\begin{center}
\begin{circuitikz}[line width=.7pt]
  \draw (0,0) node[and port] (and1) {};
\draw (and1.in 1) -- ++(-0.5,0) node[label=left:a] {};
\draw (and1.in 2) -- ++(-0.5,0) node[label=left:b] {};

\draw (and1.out) -- ++(1,0) node[nor port, anchor=in 1] (nor1) {};
\draw (nor1) ++(0,-2) node[and port] (and2) {};

\draw (and1.in 1) ++(-0.5,-1.5) node[label=left:c] (c) {};
\draw (c) -| (nor1.in 2);
\draw (c) -| (and2.in 1);

\draw (c |- and2.in 2) node[label=left:d] {} -- (and2.in 2);

\draw (nor1.out) -| ++(2,-1) node[or port, anchor=in 1] (or2) {};
\draw (and2.out) -| ++(0.3,-0.5) node[or port, anchor=in 1] (or1) {};
\draw (or1.in 2) -- ++(0,0) node[label=left:e] {};
\draw (or1.out) -| (or2.in 2);

\draw (or2.out) -- ++(0,0) node[label=right:m] {};

\draw (nor1.in 1) node[label=f] {};
\draw (nor1.in 2) node[label=left:g] {};
\draw (and2.in 1) node[label=left:h] {};
\draw (or1.in 1) node[label=left:i] {};
\draw (or2.in 1) node[label=left:j] {};
\draw (or2.in 2) node[label=left:k] {};

  % G1 in 1
  \draw[color=red, -latexslim] (and1.in 1) ++(0,0.25) -- ++(0,-0.5);
  \draw[color=red, -latexslim] (and1.in 1) ++(0.1,-0.25) -- ++(0,0.5);
  % G1 in 2
  \draw[color=red, -latexslim] (and1.in 2) ++(0,0.25) -- ++(0,-0.5);
  \draw[color=red, -latexslim] (and1.in 2) ++(0.1,-0.25) -- ++(0,0.5);
  % c
  \draw[color=red, -latexslim] (and1.in 1) ++(0,-1.25) -- ++(0,-0.5);
  \draw[color=red, -latexslim] (and1.in 1) ++(0.1,-1.75) -- ++(0,0.5);
  % G2 in 1
  \draw[color=red, -latexslim] (nor1.in 1) ++(0,0.25) -- ++(0,-0.5);
  \draw[color=red, -latexslim] (nor1.in 1) ++(0.1,-0.25) -- ++(0,0.5);
  % G2 in 2
  \draw[color=red, -latexslim] (nor1.in 2) ++(0,0.25) -- ++(0,-0.5);
  \draw[-latexslim] (nor1.in 2) ++(0.1,-0.25) -- ++(0,0.5);
  % G3 in 1
  \draw[-latexslim] (and2.in 1) ++(0,0.25) -- ++(0,-0.5);
  \draw[color=red, -latexslim] (and2.in 1) ++(0.1,-0.25) -- ++(0,0.5);
  % G3 in 2
  \draw[color=red, -latexslim] (and2.in 2) ++(0,0.25) -- ++(0,-0.5);
  \draw[color=red, -latexslim] (and2.in 2) ++(0.1,-0.25) -- ++(0,0.5);
  % G4 in 1
  \draw[-latexslim] (or1.in 1) ++(0,0.25) -- ++(0,-0.5);
  \draw[-latexslim] (or1.in 1) ++(0.1,-0.25) -- ++(0,0.5);
  % G4 in 2
  \draw[color=red, -latexslim] (or1.in 2) ++(0,0.25) -- ++(0,-0.5);
  \draw[-latexslim] (or1.in 2) ++(0.1,-0.25) -- ++(0,0.5);
  % G5 in 1
  \draw[-latexslim] (or2.in 1) ++(0,0.25) -- ++(0,-0.5);
  \draw[-latexslim] (or2.in 1) ++(0.1,-0.25) -- ++(0,0.5);
  % G5 in 2
  \draw[-latexslim] (or2.in 2) ++(0,0.25) -- ++(0,-0.5);
  \draw[-latexslim] (or2.in 2) ++(0.1,-0.25) -- ++(0,0.5);
  % G5 out
  \draw[-latexslim] (or2.out) ++(0,0.25) -- ++(0,-0.5);
  \draw[-latexslim] (or2.out) ++(0.1,-0.25) -- ++(0,0.5);

  \draw (and1) node[left=5pt] {G1};
  \draw (nor1) node[left=5pt] {G2};
  \draw (and2) node[left=5pt] {G3};
  \draw (or1) node[left=5pt] {G4};
  \draw (or2) node[left=5pt] {G5};
\end{circuitikz}
\captionof{figure}{At G2, j-sa1 dominates f-sa0 and g-sa0. g-sa1 and j-sa0 are equivalent to f-sa1.}
\end{center}

\begin{center}
\begin{circuitikz}[line width=.7pt]
  \draw (0,0) node[and port] (and1) {};
\draw (and1.in 1) -- ++(-0.5,0) node[label=left:a] {};
\draw (and1.in 2) -- ++(-0.5,0) node[label=left:b] {};

\draw (and1.out) -- ++(1,0) node[nor port, anchor=in 1] (nor1) {};
\draw (nor1) ++(0,-2) node[and port] (and2) {};

\draw (and1.in 1) ++(-0.5,-1.5) node[label=left:c] (c) {};
\draw (c) -| (nor1.in 2);
\draw (c) -| (and2.in 1);

\draw (c |- and2.in 2) node[label=left:d] {} -- (and2.in 2);

\draw (nor1.out) -| ++(2,-1) node[or port, anchor=in 1] (or2) {};
\draw (and2.out) -| ++(0.3,-0.5) node[or port, anchor=in 1] (or1) {};
\draw (or1.in 2) -- ++(0,0) node[label=left:e] {};
\draw (or1.out) -| (or2.in 2);

\draw (or2.out) -- ++(0,0) node[label=right:m] {};

\draw (nor1.in 1) node[label=f] {};
\draw (nor1.in 2) node[label=left:g] {};
\draw (and2.in 1) node[label=left:h] {};
\draw (or1.in 1) node[label=left:i] {};
\draw (or2.in 1) node[label=left:j] {};
\draw (or2.in 2) node[label=left:k] {};

  % G1 in 1
  \draw[color=red, -latexslim] (and1.in 1) ++(0,0.25) -- ++(0,-0.5);
  \draw[color=red, -latexslim] (and1.in 1) ++(0.1,-0.25) -- ++(0,0.5);
  % G1 in 2
  \draw[color=red, -latexslim] (and1.in 2) ++(0,0.25) -- ++(0,-0.5);
  \draw[color=red, -latexslim] (and1.in 2) ++(0.1,-0.25) -- ++(0,0.5);
  % c
  \draw[color=red, -latexslim] (and1.in 1) ++(0,-1.25) -- ++(0,-0.5);
  \draw[color=red, -latexslim] (and1.in 1) ++(0.1,-1.75) -- ++(0,0.5);
  % G2 in 1
  \draw[-latexslim] (nor1.in 1) ++(0,0.25) -- ++(0,-0.5);
  \draw[-latexslim] (nor1.in 1) ++(0.1,-0.25) -- ++(0,0.5);
  % G2 in 2
  \draw[color=red, -latexslim] (nor1.in 2) ++(0,0.25) -- ++(0,-0.5);
  \draw[color=red, -latexslim] (nor1.in 2) ++(0.1,-0.25) -- ++(0,0.5);
  % G3 in 1
  \draw[color=red, -latexslim] (and2.in 1) ++(0,0.25) -- ++(0,-0.5);
  \draw[color=red, -latexslim] (and2.in 1) ++(0.1,-0.25) -- ++(0,0.5);
  % G3 in 2
  \draw[color=red, -latexslim] (and2.in 2) ++(0,0.25) -- ++(0,-0.5);
  \draw[color=red, -latexslim] (and2.in 2) ++(0.1,-0.25) -- ++(0,0.5);
  % G4 in 1
  \draw[-latexslim] (or1.in 1) ++(0,0.25) -- ++(0,-0.5);
  \draw[-latexslim] (or1.in 1) ++(0.1,-0.25) -- ++(0,0.5);
  % G4 in 2
  \draw[color=red, -latexslim] (or1.in 2) ++(0,0.25) -- ++(0,-0.5);
  \draw[color=red, -latexslim] (or1.in 2) ++(0.1,-0.25) -- ++(0,0.5);
  % G5 in 1
  \draw[-latexslim] (or2.in 1) ++(0,0.25) -- ++(0,-0.5);
  \draw[-latexslim] (or2.in 1) ++(0.1,-0.25) -- ++(0,0.5);
  % G5 in 2
  \draw[-latexslim] (or2.in 2) ++(0,0.25) -- ++(0,-0.5);
  \draw[-latexslim] (or2.in 2) ++(0.1,-0.25) -- ++(0,0.5);
  % G5 out
  \draw[-latexslim] (or2.out) ++(0,0.25) -- ++(0,-0.5);
  \draw[-latexslim] (or2.out) ++(0.1,-0.25) -- ++(0,0.5);

  \draw (and1) node[left=5pt] {G1};
  \draw (nor1) node[left=5pt] {G2};
  \draw (and2) node[left=5pt] {G3};
  \draw (or1) node[left=5pt] {G4};
  \draw (or2) node[left=5pt] {G5};
\end{circuitikz}
\captionof{figure}{At G1, f-sa1 dominates a-sa1 and b-sa1. f-sa0 is equivalent to a-sa0. We've already collapsed the faults into 7 check points. For the rest of the steps, just follow problem 3(b) above.}
\end{center}


    \end{solution}

  \end{parts}
\end{questions}
\end{document}

