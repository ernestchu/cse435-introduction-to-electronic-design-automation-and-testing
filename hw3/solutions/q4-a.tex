In the figure below, the red circles indicate the missing crosspoint faults, and the upward arrows indicate the s-a-1 faults. The missing crosspoint in the AND plane causes growth fault, i.e., the lines having crosspoint with C will no longer depend on C.
\begin{center}
  \begin{minipage}{0.4\linewidth}
    \raggedleft
    \begin{circuitikz}[line width=.7pt]
      \ctikzset{logic ports=ieee}
    
      % A & A negate
\draw (0,0) node[label=A] (A) {} -- ++(0,-4.5);
\draw (0.5,-1) node[not port, rotate=-90, scale=0.4, circuitikz/ieeestd ports/not radius=.25] (A_neg) {};
\draw (A_neg.in) |- ++(-0.5,0.25) to[short, *-] ++(0,0);
\draw (A_neg.out) -- (0.5,-4.5);

% B & B negate
\draw (1,0) node[label=B] (B) {} -- ++(0,-4.5);
\draw (1.5,-1) node[not port, rotate=-90, scale=0.4, circuitikz/ieeestd ports/not radius=.25] (B_neg) {};
\draw (B_neg.in) |- ++(-0.5,0.25) to[short, *-] ++(0,0);
\draw (B_neg.out) -- (1.5,-4.5);

% C & C negate
\draw (2,0) node[label=C] (C) {} -- ++(0,-4.5);
\draw (2.5,-1) node[not port, rotate=-90, scale=0.4, circuitikz/ieeestd ports/not radius=.25] (C_neg) {};
\draw (C_neg.in) |- ++(-0.5,0.25) to[short, *-] ++(0,0);
\draw (C_neg.out) -- (2.5,-4.5);

% f1 & f2
\draw (4.0,0) node[label=f1] (f1) {} -- ++(0,-4.5);
\draw (4.5,0) node[label=f2] (f2) {} -- ++(0,-4.5);

% horizontal lines
\draw (-0.5,-1.5) -- ++(5.5,0);
\draw (-0.5,-2.0) -- ++(5.5,0);
\draw (-0.5,-2.5) -- ++(5.5,0);
\draw (-0.5,-3.0) -- ++(5.5,0);
\draw (-0.5,-3.5) -- ++(5.5,0);
\draw (-0.5,-4.0) -- ++(5.5,0);

% and plane
% AB
\draw (0,-1.5) to[short, *-*] ++(1,0);
% BC
\draw (1,-2.0) to[short, *-*] ++(1,0);
% CA
\draw (0,-2.5) to[short, *-*] ++(2,0);
% A_negB_neg
\draw (0.5,-3.0) to[short, *-*] ++(1,0);
% B_negC_neg
\draw (1.5,-3.5) to[short, *-*] ++(1,0);
% C_negA_neg
\draw (0.5,-4.0) to[short, *-*] ++(2,0);

    
      % f1 or plane
      \draw (4,-1.5) node[label=center:$\times$] {};
      \draw (4,-2.0) node[label=center:$\times$] {};
      \draw (4,-2.5) node[label=center:$\times$] {};
      \draw (4,-3.0) node[label=center:$\times$] {};
      \draw (4,-3.5) node[label=center:$\times$] {};
      \draw (4,-4.0) node[label=center:$\times$] {};
    
      % f2 or plane
      \draw (4.5,-1.5) node[label=center:$\times$] {};
      \draw (4.5,-2.0) node[label=center:$\times$] {};
      \draw (4.5,-2.5) node[label=center:$\times$] {};
    
      % fault
      \draw[color=red] (2,-1.5) circle[radius=6pt] {};
      \draw[color=red] (2,-2.0) circle[radius=6pt] {};
      \draw[color=red] (2,-2.5) circle[radius=6pt] {};
      \draw[color=red] (2,-3.0) circle[radius=6pt] {};
      \draw[color=red] (2,-3.5) circle[radius=6pt] {};
      \draw[color=red] (2,-4.0) circle[radius=6pt] {};
      
    \end{circuitikz}
  \end{minipage}
  \hfill
  \begin{minipage}{0.5\linewidth}
    \begin{circuitikz}[line width=.7pt]
      \ctikzset{logic ports=ieee}

      % A & A negate
\draw (0,0) node[label=A] (A) {} -- ++(0,-4.5);
\draw (0.5,-1) node[not port, rotate=-90, scale=0.4, circuitikz/ieeestd ports/not radius=.25] (A_neg) {};
\draw (A_neg.in) |- ++(-0.5,0.25) to[short, *-] ++(0,0);
\draw (A_neg.out) -- (0.5,-4.5);

% B & B negate
\draw (1,0) node[label=B] (B) {} -- ++(0,-4.5);
\draw (1.5,-1) node[not port, rotate=-90, scale=0.4, circuitikz/ieeestd ports/not radius=.25] (B_neg) {};
\draw (B_neg.in) |- ++(-0.5,0.25) to[short, *-] ++(0,0);
\draw (B_neg.out) -- (1.5,-4.5);

% C & C negate
\draw (2,0) node[label=C] (C) {} -- ++(0,-4.5);
\draw (2.5,-1) node[not port, rotate=-90, scale=0.4, circuitikz/ieeestd ports/not radius=.25] (C_neg) {};
\draw (C_neg.in) |- ++(-0.5,0.25) to[short, *-] ++(0,0);
\draw (C_neg.out) -- (2.5,-4.5);

% horizontal lines
\draw (-0.5,-1.5) -- ++(3.25,0) -- ++(0,0) node[and port, anchor=in 1, scale=0.35715] (p1) {};
\draw (-0.5,-1.7) -- ++(3.25,0) -- (p1.in 2);
\draw (-0.5,-2.0) -- ++(3.25,0) -- ++(0,0) node[and port, anchor=in 1, scale=0.35715] (p2) {};
\draw (-0.5,-2.2) -- ++(3.25,0);
\draw (-0.5,-2.5) -- ++(3.25,0) -- ++(0,0) node[and port, anchor=in 1, scale=0.35715] (p3) {};
\draw (-0.5,-2.7) -- ++(3.25,0);
\draw (-0.5,-3.0) -- ++(3.25,0) -- ++(0,0) node[and port, anchor=in 1, scale=0.35715] (p4) {};
\draw (-0.5,-3.2) -- ++(3.25,0);
\draw (-0.5,-3.5) -- ++(3.25,0) -- ++(0,0) node[and port, anchor=in 1, scale=0.35715] (p5) {};
\draw (-0.5,-3.7) -- ++(3.25,0);
\draw (-0.5,-4.0) -- ++(3.25,0) -- ++(0,0) node[and port, anchor=in 1, scale=0.35715] (p6) {};
\draw (-0.5,-4.2) -- ++(3.25,0);

% and plane
% AB
\draw (0,-1.5)  to[short, *-] ++(0,0);
\draw (1,-1.7) to[short, *-] ++(0,0);
% BC
\draw (1,-2.0)  to[short, *-] ++(0,0);
\draw (2,-2.2) to[short, *-] ++(0,0);
% CA
\draw (0,-2.5)  to[short, *-] ++(0,0);
\draw (2,-2.7) to[short, *-] ++(0,0);
% A_negB_neg
\draw (0.5,-3.0)  to[short, *-] ++(0,0);
\draw (1.5,-3.2) to[short, *-] ++(0,0);
% B_negC_neg
\draw (1.5,-3.5)  to[short, *-] ++(0,0);
\draw (2.5,-3.7) to[short, *-] ++(0,0);
% C_negA_neg
\draw (0.5,-4.0)  to[short, *-] ++(0,0);
\draw (2.5,-4.2) to[short, *-] ++(0,0);

    
      % f1
      \draw (5,-2.85) node[or port, scale=0.5, number inputs=6] (f1) {};
      \draw (p1.out) -- ++(0.2,0) |- (f1.in 1);
      \draw (p2.out) -- ++(0.1,0) |- (f1.in 2);
      \draw (p3.out) |- (f1.in 3);
      \draw (p4.out) |- (f1.in 4);
      \draw (p5.out) -- ++(0.1,0) |- (f1.in 5);
      \draw (p6.out) -- ++(0.2,0) |- (f1.in 6);
      \draw (f1.out) -- ++(0,0) node[label=right:f1] {};

      % f2
      \draw (5,-2) node[or port, scale=0.5, number inputs=3] (f2) {};

      \draw (f1.in 1) to[short,*-] (f2.in 3);

      \draw (f2.in 2) ++(-0.2,0) node (f2_in2_shifted) {} -- (f2.in 2);
      \draw (f1.in 2) ++(-0.2,0) to[short,*-] (f2_in2_shifted.center);

      \draw (f2.in 1) ++(-0.4,0) node (f2_in1_shifted) {} -- (f2.in 1);
      \draw (f1.in 3) ++(-0.4,0) to[short,*-] (f2_in1_shifted.center);

      \draw (f2.out) -- ++(0,0) node[label=right:f2] {};
    
      % fault
      \draw[color=red, -latexslim] (p2.in 2) ++(-0.05,-0.15) -- ++(0,0.3);
      \draw[color=red, -latexslim] (p3.in 2) ++(-0.05,-0.15) -- ++(0,0.3);
      \draw[color=red] (3.25,-1.5) node[label=s-a-1] {};
    \end{circuitikz}
  \end{minipage}
\end{center}
