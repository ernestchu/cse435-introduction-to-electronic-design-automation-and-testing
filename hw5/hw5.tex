\documentclass[12pt,answers]{exam}
\usepackage{fontspec,graphicx,circuitikz,amsmath,caption}
\setmainfont{Times New Roman}

\headheight 8pt \headsep 20pt \footskip 30pt
\textheight 9in \textwidth 6.5in
\oddsidemargin 0in \evensidemargin 0in
\topmargin -.35in

\begin{document}
\begin{center}
\LARGE CSE435 Introduction to EDA \& Testing - Spring 2022 \\
\Large Homework Assignment \#5 \\
\Large Shao-Hsuan Chu - B073040018 \\
\end{center}
\bigskip

\begin{questions}
  \question (20\%) A circuit has the truth table of Table 1a. When there is a fault (faults) on the circuit, the faulty truth table becomes Table 1b. Try to derive tests to detect the fault (faults).
  The defect level $\textbf{DL}$ can be obtained by $\textbf{DL} = 1 - \textbf{Y} ^ {1 - \textbf{T}}$, where $\textbf{Y}$ is the yield, indicating the manufacturing capability, and $\textbf{T}$ is the fault coverage, indicating the testing capability.
\begin{align*} 
0.001                             &\geq 1 - 0.9 ^ {1 - \textbf{T}} \\
0.9 ^ {1 - \textbf{T}}            &\geq 1 - 0.001 \\
0.9 ^ {1 - \textbf{T}}            &\geq 0.999 \\
\log_{0.9} 0.9 ^ {1 - \textbf{T}} &\leq \log_{0.9} 0.999 \\
1 - \textbf{T}                    &\leq 0.0095 \\
\textbf{T}                        &\geq 0.9905 = 99.05\%
\end{align*}
\textbf{Answer:} The fault coverage $\textbf{Y}$ must be at least 99.05\%

  \begin{solution}
  \end{solution}

  \question (80\%) Generate a test for the fault f-sa1 in Figure 1 by the following FOUR methods. Be sure to give the \textbf{key steps to show the features of every algorithm}, and also \textbf{draw the decision trees} for each case.
  \begin{parts}
    \part (20\%) Use the \textbf{Boolean difference method} to derive all the test patterns to detect the fault f-sa1.
    \part (20\%) Generate a test for the fault f-sa1 by using \textbf{D-algorithm}.
    \part (20\%) Generate a test for the fault f-sa1 by using \textbf{9-V Algorithm}.
    \part (20\%) Generate a test for the fault f-sa1 by using \textbf{PODEM algorithm}.
  \end{parts}
  The test pattern can be given as the table below. Each row is a clock tick. Assume $\text{Q}_\text{n}$ = 0 at the initial tick. {\color{red} (Todo: How to make such assumption?)} For all the following ticks, $\text{Q}_\text{n}$ is the $\text{Q}_\text{n+1}$ from the previous tick.
\begin{center}
  \begin{tabular}{c | c | c}
    T & $\text{Q}_\text{n+1}$ & Functions \\
    \hline
    0 & 0 & hold 0 \\
    1 & 1 & toggle to 1 \\
    0 & 1 & hold 1 \\
    1 & 0 & toggle to 0 \\
  \end{tabular}
\end{center}
If output does not meet the expectation, a functional fault is detected in the F/F. One of the hold or toggle to 0/1 functions could not behave properly.
 


\end{questions}
\end{document}

