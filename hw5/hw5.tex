\documentclass[12pt,answers]{exam}
\usepackage{fontspec,graphicx,circuitikz,amsmath,caption,tikz,ctable}
\setmainfont{Times New Roman}

\headheight 8pt \headsep 20pt \footskip 30pt
\textheight 9in \textwidth 6.5in
\oddsidemargin 0in \evensidemargin 0in
\topmargin -.35in

\begin{document}
\begin{center}
\LARGE CSE435 Introduction to EDA \& Testing - Spring 2022 \\
\Large Homework Assignment \#5 \\
\Large Shao-Hsuan Chu - B073040018 \\
\end{center}
\bigskip

\begin{questions}
  \question (20\%) A circuit has the truth table of Table 1. When there is a fault (faults) on the circuit, the faulty truth table becomes Table 2. Try to derive tests to detect the fault (faults).
  The defect level $\textbf{DL}$ can be obtained by $\textbf{DL} = 1 - \textbf{Y} ^ {1 - \textbf{T}}$, where $\textbf{Y}$ is the yield, indicating the manufacturing capability, and $\textbf{T}$ is the fault coverage, indicating the testing capability.
\begin{align*} 
0.001                             &\geq 1 - 0.9 ^ {1 - \textbf{T}} \\
0.9 ^ {1 - \textbf{T}}            &\geq 1 - 0.001 \\
0.9 ^ {1 - \textbf{T}}            &\geq 0.999 \\
\log_{0.9} 0.9 ^ {1 - \textbf{T}} &\leq \log_{0.9} 0.999 \\
1 - \textbf{T}                    &\leq 0.0095 \\
\textbf{T}                        &\geq 0.9905 = 99.05\%
\end{align*}
\textbf{Answer:} The fault coverage $\textbf{Y}$ must be at least 99.05\%

  \begin{solution}
    The defect level $\textbf{DL}$ can be obtained by $\textbf{DL} = 1 - \textbf{Y} ^ {1 - \textbf{T}}$, where $\textbf{Y}$ is the yield, indicating the manufacturing capability, and $\textbf{T}$ is the fault coverage, indicating the testing capability.
\begin{align*} 
0.001                             &\geq 1 - 0.9 ^ {1 - \textbf{T}} \\
0.9 ^ {1 - \textbf{T}}            &\geq 1 - 0.001 \\
0.9 ^ {1 - \textbf{T}}            &\geq 0.999 \\
\log_{0.9} 0.9 ^ {1 - \textbf{T}} &\leq \log_{0.9} 0.999 \\
1 - \textbf{T}                    &\leq 0.0095 \\
\textbf{T}                        &\geq 0.9905 = 99.05\%
\end{align*}
\textbf{Answer:} The fault coverage $\textbf{Y}$ must be at least 99.05\%

  \end{solution}

  \question (80\%) Generate a test for the fault f-sa1 in Figure 1 by the following FOUR methods. Be sure to give the \textbf{key steps to show the features of every algorithm}, and also \textbf{draw the decision trees} for each case.
  The test pattern can be given as the table below. Each row is a clock tick. Assume $\text{Q}_\text{n}$ = 0 at the initial tick. {\color{red} (Todo: How to make such assumption?)} For all the following ticks, $\text{Q}_\text{n}$ is the $\text{Q}_\text{n+1}$ from the previous tick.
\begin{center}
  \begin{tabular}{c | c | c}
    T & $\text{Q}_\text{n+1}$ & Functions \\
    \hline
    0 & 0 & hold 0 \\
    1 & 1 & toggle to 1 \\
    0 & 1 & hold 1 \\
    1 & 0 & toggle to 0 \\
  \end{tabular}
\end{center}
If output does not meet the expectation, a functional fault is detected in the F/F. One of the hold or toggle to 0/1 functions could not behave properly.
 
  \begin{parts}
    \part (20\%) Use the \textbf{Boolean difference method} to derive all the test patterns to detect the fault f-sa1.
    \begin{solution}
      The Lost Revenue $\textbf{LR}$ equals to the absolute difference between the Total Expected Revenue $\textbf{TER}$ and the Total Actual Revenue $\textbf{TAR}$. Denote the market growth rates as $r$, the Lost Revenue can be obtained by
\begin{align*}
\textbf{TER} &= \frac{1}{2} (2w) (w)r = (w^2) r \\
\textbf{TAR} &= \frac{1}{2} (2w - d) (w-d)r = (w^2 - \frac{3}{2}wd + \frac{1}{2}d^2) r \\
\textbf{LR}  &= \textbf{TER} - \textbf{TAR} \\
\textbf{LR}  &= (\frac{3}{2}wd - \frac{1}{2}d^2) r \\
\textbf{LR}  &= \textbf{TER} \times \frac{(\frac{3}{2}wd - \frac{1}{2}d^2) r}{\textbf{TER}} \\
\textbf{LR}  &= \textbf{TER} \times \frac{\frac{3}{2}wd - \frac{1}{2}d^2}{w^2} \\
\end{align*}
\textbf{Answer:} $(\frac{3}{2}wd - \frac{1}{2}d^2) / w^2$

    \end{solution}
    \part (20\%) Generate a test for the fault f-sa1 by using \textbf{D-algorithm}.
    \begin{solution}
      
\begin{tabular}{rlcc|c|c|c|c|c|c|}
  \specialrule{.1em}{.05em}{.05em} 
  \multirow{2}{*}{Step} &
  \multirow{2}{*}{Comments} &
  \multicolumn{7}{c}{Cube} \\
  \cline{3-10}
  && \multicolumn{1}{|c|}{a} & b & c & d & e & f & g & h \\
  \specialrule{.1em}{.05em}{.05em} 
  1 &
  Activate f-sa1. TC(0) = PDCF &
% \multicolumn{1}{|c|}{a} & b & c & d & e & f & g & h \\
  \multicolumn{1}{|c|}{ } &   & 1 &   & 1 & D'&   &   \\
  \hline
  2 &
  Backward implication: SC\textsubscript{G1} &
% \multicolumn{1}{|c|}{a} & b & c & d & e & f & g & h \\
  \multicolumn{1}{|c|}{\textbf{0}} & \textbf{0} &   &   & 1 &  &   &   \\
  & TC(1) = TC(0) $\cap$ SC\textsubscript{G1} &
% \multicolumn{1}{|c|}{a} & b & c & d & e & f & g & h \\
  \multicolumn{1}{|c|}{0} & 0 & 1 &   & 1 & D'&   &   \\
  \hline
  3 &
  Propagate fault. D-drive = G4. PDC\textsubscript{G4} &
% \multicolumn{1}{|c|}{a} & b & c & d & e & f & g & h \\
  \multicolumn{1}{|c|}{ } &   &   &   &   & D'& 0 & D \\
  & TC(2) = TC(1) $\cap$ PDC\textsubscript{G4} &
% \multicolumn{1}{|c|}{a} & b & c & d & e & f & g & h \\
  \multicolumn{1}{|c|}{0} & 0 & 1 &   & 1 & D'& 0 & D \\
  \hline
  4 &
  Backward implication: SC\textsubscript{G3} &
% \multicolumn{1}{|c|}{a} & b & c & d & e & f & g & h \\
  \multicolumn{1}{|c|}{ } &   & 1 & \textbf{0} &   &   & 0 &   \\
  & TC(3) = TC(2) $\cap$ SC\textsubscript{G3} &
% \multicolumn{1}{|c|}{a} & b & c & d & e & f & g & h \\
  \multicolumn{1}{|c|}{0} & 0 & 1 & 0 & 1 & D'& 0 & D \\
  \hline
  5 &
  \multicolumn{9}{l}{No justification needed. Test generated.} \\
  \specialrule{.1em}{.05em}{.05em} 
\end{tabular}

\textbf{Answer:} (a, b, c, d) = (0, 0, 1, 0).


    \end{solution}
    \part (20\%) Generate a test for the fault f-sa1 by using \textbf{9-V Algorithm}.
    \begin{solution}
      The acronyms in Problem 2(b) are also used in the 9-V algorithm procedure, shown in Table 5.
\begin{center}
\begin{tabular}{rlcc|c|c|c|c|c|c|}
  \specialrule{.1em}{.05em}{.05em} 
  \multirow{2}{*}{Step} &
  \multirow{2}{*}{Comments} &
  \multicolumn{7}{c}{Cube} \\
  \cline{3-10}
  && \multicolumn{1}{|c|}{a} & b & c & d & e & f & g & h \\
  \specialrule{.1em}{.05em}{.05em} 
  1 &
  Activate f-sa1. TC(0) = PDCF &
% \multicolumn{1}{|c|}{a} & b & c & d & e & f & g & h \\
  \multicolumn{1}{|c|}{ } &   & 1 &   & 1 & D'&   &   \\
  \hline
  2 &
  Backward implication: SC\textsubscript{G1} &
% \multicolumn{1}{|c|}{a} & b & c & d & e & f & g & h \\
  \multicolumn{1}{|c|}{\textbf{0}} & \textbf{0} &   &   & 1 &  &   &   \\
  & Forward implication: SC\textsubscript{G4} &
% \multicolumn{1}{|c|}{a} & b & c & d & e & f & g & h \\
  \multicolumn{1}{|c|}{ } &   &   &   &   & D'&   & \textbf{0/D} \\
  & TC(1) = TC(0) $\cap$ SC\textsubscript{G1} $\cap$ SC\textsubscript{G4} &
% \multicolumn{1}{|c|}{a} & b & c & d & e & f & g & h \\
  \multicolumn{1}{|c|}{0} & 0 & 1 &   & 1 & D'&   & 0/D \\
  \hline
  3 &
  Propagate fault. D-drive = G4. PDC\textsubscript{G4} &
% \multicolumn{1}{|c|}{a} & b & c & d & e & f & g & h \\
  \multicolumn{1}{|c|}{ } &   &   &   &   & D'& 0 & D \\
  & TC(2) = TC(1) $\cap$ PDC\textsubscript{G4} &
% \multicolumn{1}{|c|}{a} & b & c & d & e & f & g & h \\
  \multicolumn{1}{|c|}{0} & 0 & 1 &   & 1 & D'& 0 & D \\
  \hline
  4 &
  Backward implication: SC\textsubscript{G3} &
% \multicolumn{1}{|c|}{a} & b & c & d & e & f & g & h \\
  \multicolumn{1}{|c|}{ } &   & 1 & \textbf{0} &   &   & 0 &   \\
  & TC(3) = TC(2) $\cap$ SC\textsubscript{G3} &
% \multicolumn{1}{|c|}{a} & b & c & d & e & f & g & h \\
  \multicolumn{1}{|c|}{0} & 0 & 1 & 0 & 1 & D'& 0 & D \\
  \hline
  5 &
  \multicolumn{9}{l}{No justification needed. Test generated, (a, b, c, d) = (0, 0, 1, 0)} \\
  \specialrule{.1em}{.05em}{.05em} 
\end{tabular}
\captionof{table}{The procedure for 9-V algorithm}
\end{center}

\textbf{Answer:} (a, b, c, d) = (0, 0, 1, 0).


    \end{solution}
    \part (20\%) Generate a test for the fault f-sa1 by using \textbf{PODEM algorithm}.
    \begin{solution}
      \begin{center} \begin{circuitikz}[line width=.7pt]
  \draw (0,0) node[and port] (and1) {};
\draw (and1.in 1) -- ++(-0.5,0) node[label=left:a] {};
\draw (and1.in 2) -- ++(-0.5,0) node[label=left:b] {};

\draw (and1.out) -- ++(1,0) node[nor port, anchor=in 1] (nor1) {};
\draw (nor1) ++(0,-2) node[and port] (and2) {};

\draw (and1.in 1) ++(-0.5,-1.5) node[label=left:c] (c) {};
\draw (c) -| (nor1.in 2);
\draw (c) -| (and2.in 1);

\draw (c |- and2.in 2) node[label=left:d] {} -- (and2.in 2);

\draw (nor1.out) -| ++(2,-1) node[or port, anchor=in 1] (or2) {};
\draw (and2.out) -| ++(0.3,-0.5) node[or port, anchor=in 1] (or1) {};
\draw (or1.in 2) -- ++(0,0) node[label=left:e] {};
\draw (or1.out) -| (or2.in 2);

\draw (or2.out) -- ++(0,0) node[label=right:m] {};

\draw (nor1.in 1) node[label=f] {};
\draw (nor1.in 2) node[label=left:g] {};
\draw (and2.in 1) node[label=left:h] {};
\draw (or1.in 1) node[label=left:i] {};
\draw (or2.in 1) node[label=left:j] {};
\draw (or2.in 2) node[label=left:k] {};

  \draw (G1.in 1) node[label=left:a 1] (a) {};
  \draw (G1.in 2) node[label=left:b$\times$] {};
  \draw (a |- G4) node[label=left:c$\times$] {} to[short, -*] (G2.in 2 |- G4);
  \draw (a |- G3.in 2) node[label=left:d$\times$] {} -- (G3.in 2);
  \draw (G2.in 1) node[label=e$\times$] {};
  \draw (G2.out) node[label=right:f 0/1] {};
  \draw (G3.out) node[label=right:g$\times$] {};
  \draw (G4.out) node[label=right:h$\times$] {};
  \draw[color=red, line width=0.75mm] (G4.in 1) -| (G2.out) |- (G1.out) |- (G1.in 1);
\end{circuitikz}
\captionof{figure}{Activate f-sa1, so f = 0. Objective = (f, 0), inversion parity = even. Therefore, a = f = 0.}
\end{center}

\begin{center} \begin{circuitikz}[line width=.7pt]
  \draw (0,0) node[and port] (and1) {};
\draw (and1.in 1) -- ++(-0.5,0) node[label=left:a] {};
\draw (and1.in 2) -- ++(-0.5,0) node[label=left:b] {};

\draw (and1.out) -- ++(1,0) node[nor port, anchor=in 1] (nor1) {};
\draw (nor1) ++(0,-2) node[and port] (and2) {};

\draw (and1.in 1) ++(-0.5,-1.5) node[label=left:c] (c) {};
\draw (c) -| (nor1.in 2);
\draw (c) -| (and2.in 1);

\draw (c |- and2.in 2) node[label=left:d] {} -- (and2.in 2);

\draw (nor1.out) -| ++(2,-1) node[or port, anchor=in 1] (or2) {};
\draw (and2.out) -| ++(0.3,-0.5) node[or port, anchor=in 1] (or1) {};
\draw (or1.in 2) -- ++(0,0) node[label=left:e] {};
\draw (or1.out) -| (or2.in 2);

\draw (or2.out) -- ++(0,0) node[label=right:m] {};

\draw (nor1.in 1) node[label=f] {};
\draw (nor1.in 2) node[label=left:g] {};
\draw (and2.in 1) node[label=left:h] {};
\draw (or1.in 1) node[label=left:i] {};
\draw (or2.in 1) node[label=left:j] {};
\draw (or2.in 2) node[label=left:k] {};

  \draw (G1.in 1) node[label=left:a 0] (a) {};
  \draw (G1.in 2) node[label=left:b 0] {};
  \draw (a |- G4) node[label=left:c$\times$] {} to[short, -*] (G2.in 2 |- G4);
  \draw (a |- G3.in 2) node[label=left:d$\times$] {} -- (G3.in 2);
  \draw (G2.in 1) node[label=e 1] {};
  \draw (G2.out) node[label=right:f 0/1] {};
  \draw (G3.out) node[label=right:g$\times$] {};
  \draw (G4.out) node[label=right:h$\times$] {};
  \draw[color=red, line width=0.75mm] (G4.in 1) -| (G2.out) |- (G1.out) |- (G1.in 2);
\end{circuitikz}
\captionof{figure}{Objective = (f, 0), inversion parity = even. Therefore, b = f = 0, e = 1.}
\end{center}

\begin{center} \begin{circuitikz}[line width=.7pt]
  \draw (0,0) node[and port] (and1) {};
\draw (and1.in 1) -- ++(-0.5,0) node[label=left:a] {};
\draw (and1.in 2) -- ++(-0.5,0) node[label=left:b] {};

\draw (and1.out) -- ++(1,0) node[nor port, anchor=in 1] (nor1) {};
\draw (nor1) ++(0,-2) node[and port] (and2) {};

\draw (and1.in 1) ++(-0.5,-1.5) node[label=left:c] (c) {};
\draw (c) -| (nor1.in 2);
\draw (c) -| (and2.in 1);

\draw (c |- and2.in 2) node[label=left:d] {} -- (and2.in 2);

\draw (nor1.out) -| ++(2,-1) node[or port, anchor=in 1] (or2) {};
\draw (and2.out) -| ++(0.3,-0.5) node[or port, anchor=in 1] (or1) {};
\draw (or1.in 2) -- ++(0,0) node[label=left:e] {};
\draw (or1.out) -| (or2.in 2);

\draw (or2.out) -- ++(0,0) node[label=right:m] {};

\draw (nor1.in 1) node[label=f] {};
\draw (nor1.in 2) node[label=left:g] {};
\draw (and2.in 1) node[label=left:h] {};
\draw (or1.in 1) node[label=left:i] {};
\draw (or2.in 1) node[label=left:j] {};
\draw (or2.in 2) node[label=left:k] {};

  \draw (G1.in 1) node[label=left:a 0] (a) {};
  \draw (G1.in 2) node[label=left:b 0] {};
  \draw (a |- G4) node[label=left:c 1] (c) {} to[short, -*] (G2.in 2 |- G4);
  \draw (a |- G3.in 2) node[label=left:d$\times$] {} -- (G3.in 2);
  \draw (G2.in 1) node[label=e 1] {};
  \draw (G2.out) node[label=right:f 0/1] {};
  \draw (G3.out) node[label=right:g$\times$] {};
  \draw (G4.out) node[label=right:h$\times$] {};
  \draw[color=red, line width=0.75mm] (G4.in 1) -| (G2.out) -| (G2.in 2) |- (c);
\end{circuitikz}
\captionof{figure}{Objective = (f, 0), inversion parity = odd. Therefore, c = f' = 1}
\end{center}

\begin{center} \begin{circuitikz}[line width=.7pt]
  \draw (0,0) node[and port] (and1) {};
\draw (and1.in 1) -- ++(-0.5,0) node[label=left:a] {};
\draw (and1.in 2) -- ++(-0.5,0) node[label=left:b] {};

\draw (and1.out) -- ++(1,0) node[nor port, anchor=in 1] (nor1) {};
\draw (nor1) ++(0,-2) node[and port] (and2) {};

\draw (and1.in 1) ++(-0.5,-1.5) node[label=left:c] (c) {};
\draw (c) -| (nor1.in 2);
\draw (c) -| (and2.in 1);

\draw (c |- and2.in 2) node[label=left:d] {} -- (and2.in 2);

\draw (nor1.out) -| ++(2,-1) node[or port, anchor=in 1] (or2) {};
\draw (and2.out) -| ++(0.3,-0.5) node[or port, anchor=in 1] (or1) {};
\draw (or1.in 2) -- ++(0,0) node[label=left:e] {};
\draw (or1.out) -| (or2.in 2);

\draw (or2.out) -- ++(0,0) node[label=right:m] {};

\draw (nor1.in 1) node[label=f] {};
\draw (nor1.in 2) node[label=left:g] {};
\draw (and2.in 1) node[label=left:h] {};
\draw (or1.in 1) node[label=left:i] {};
\draw (or2.in 1) node[label=left:j] {};
\draw (or2.in 2) node[label=left:k] {};

  \draw (G1.in 1) node[label=left:a 0] (a) {};
  \draw (G1.in 2) node[label=left:b 0] {};
  \draw (a |- G4) node[label=left:c 1] (c) {} to[short, -*] (G2.in 2 |- G4);
  \draw (a |- G3.in 2) node[label=left:d 0] (d) {} -- (G3.in 2);
  \draw (G2.in 1) node[label=e 1] {};
  \draw (G2.out) node[label=right:f 0/1] {};
  \draw (G3.out) node[label=right:g 1] {};
  \draw (G4.out) node[label=right:h 0/1] {};
  \draw[color=red, line width=0.75mm] (G4.in 2) -| (G3.out) |- (G3.in 2) |- (d);
\end{circuitikz}
\captionof{figure}{To propagate to h, g = 1. Objective = (g, 1), inversion parity = odd. Therefore, d = g' = 0. f-sa1 is now observable at h, thus complete.}
\end{center}

\textbf{Answer:} (a, b, c, d) = (0, 0, 1, 0).


    \end{solution}
  \end{parts}


\end{questions}
\end{document}

