\documentclass[12pt,answers]{exam}
\usepackage{fontspec,graphicx,circuitikz,amsmath,caption}
\setmainfont{Times New Roman}

\headheight 8pt \headsep 20pt \footskip 30pt
\textheight 9in \textwidth 6.5in
\oddsidemargin 0in \evensidemargin 0in
\topmargin -.35in

\begin{document}
\begin{center}
\LARGE CSE435 Introduction to EDA \& Testing - Spring 2022 \\
\Large Homework Assignment \#5 \\
\Large Shao-Hsuan Chu - B073040018 \\
\end{center}
\bigskip

\begin{questions}
  \question (20\%) A circuit has the truth table of Table 1a. When there is a fault (faults) on the circuit, the faulty truth table becomes Table 1b. Try to derive tests to detect the fault (faults).
  The test pattern can be given as the table below. Each row is a clock tick, and at the initial tick, $\text{Q}_\text{n}$ is in the don't-care condition. For all the following ticks, $\text{Q}_\text{n}$ is the $\text{Q}_\text{n+1}$ from the previous tick.
\begin{center}
  \begin{tabular}{c | c | c}
    D & $\text{Q}_\text{n+1}$ & Functions \\
    \hline
    0 & 0 & set 0 (with $\text{Q}_\text{n}$= $\times$) \\
    0 & 0 & set 0 (with $\text{Q}_\text{n}$= 0) \\
    1 & 1 & set 1 (with $\text{Q}_\text{n}$= 0) \\
    1 & 1 & set 1 (with $\text{Q}_\text{n}$= 1) \\
    0 & 0 & set 0 (with $\text{Q}_\text{n}$= 1) \\
  \end{tabular}
\end{center}
If output does not meet the expectation, a functional fault is detected in the F/F. One of the set-to-0 or set-to-1 functions could not behave properly.

  \begin{solution}
  \end{solution}

  \question (80\%) Generate a test for the fault f-sa1 in Figure 1 by the following FOUR methods. Be sure to give the \textbf{key steps to show the features of every algorithm}, and also \textbf{draw the decision trees} for each case.
  \begin{parts}
    \part (20\%) Use the \textbf{Boolean difference method} to derive all the test patterns to detect the fault f-sa1.
    \part (20\%) Generate a test for the fault f-sa1 by using \textbf{D-algorithm}.
    \part (20\%) Generate a test for the fault f-sa1 by using \textbf{9-V Algorithm}.
    \part (20\%) Generate a test for the fault f-sa1 by using \textbf{PODEM algorithm}.
  \end{parts}
  When $R_C$ is abnormally high, the output would become slow to charge, which leads to the gate-delay fault. The time required to charge the capacitor $\tau$, which can be formulated as $\tau = RC$, where $R$ is the resistance, and $C$ is the capacitance, thus $\tau \propto R$. The delay is increased by 100 times as well.

To test the charging delay fault, we should first clear the output by setting (A, B) to (0, 1), (1, 0), or (1, 1). At the second time frame, we set (A, B) to (0, 0) to charge the circuit and observe the output to test if it produces significant delay.

\textbf{Answer:} At time frame 1, (A, B) = (0, 1), (1, 0), or (1, 1). At time frame 2, (A, B) = (0, 0).
 


\end{questions}
\end{document}

