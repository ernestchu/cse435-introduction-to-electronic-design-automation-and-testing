\begin{center} \begin{circuitikz}[line width=.7pt]
  \draw (0,0) node[nor port] (G1) {};
\draw (G1.out) -- ++(0,0) node[nand port, anchor=in 1] (G2) {};
\draw (G2.in 2) -- ++(0,-1) node[xnor port, anchor=in 1] (G3) {};
\draw (G2.out) |- ++(0.5,-0.5) node[nor port, anchor=in 1] (G4) {};
\draw (G3.out) |- (G4.in 2);

\draw (G1) node[left=6pt] {G1};
\draw (G2) node[left=6pt] {G2};
\draw (G3) node[left=6pt] {G3};
\draw (G4) node[left=6pt] {G4};

  \draw (G1.in 1) node[label=left:a 1] (a) {};
  \draw (G1.in 2) node[label=left:b$\times$] {};
  \draw (a |- G4) node[label=left:c$\times$] {} to[short, -*] (G2.in 2 |- G4);
  \draw (a |- G3.in 2) node[label=left:d$\times$] {} -- (G3.in 2);
  \draw (G2.in 1) node[label=e$\times$] {};
  \draw (G2.out) node[label=right:f 0/1] {};
  \draw (G3.out) node[label=right:g$\times$] {};
  \draw (G4.out) node[label=right:h$\times$] {};
  \draw[color=red, line width=0.75mm] (G4.in 1) -| (G2.out) |- (G1.out) |- (G1.in 1);
\end{circuitikz}
\captionof{figure}{Activate f-sa1, so f = 0. Objective = (f, 0), inversion parity = even. Therefore, a = f = 0.}
\end{center}

\begin{center} \begin{circuitikz}[line width=.7pt]
  \draw (0,0) node[nor port] (G1) {};
\draw (G1.out) -- ++(0,0) node[nand port, anchor=in 1] (G2) {};
\draw (G2.in 2) -- ++(0,-1) node[xnor port, anchor=in 1] (G3) {};
\draw (G2.out) |- ++(0.5,-0.5) node[nor port, anchor=in 1] (G4) {};
\draw (G3.out) |- (G4.in 2);

\draw (G1) node[left=6pt] {G1};
\draw (G2) node[left=6pt] {G2};
\draw (G3) node[left=6pt] {G3};
\draw (G4) node[left=6pt] {G4};

  \draw (G1.in 1) node[label=left:a 0] (a) {};
  \draw (G1.in 2) node[label=left:b 0] {};
  \draw (a |- G4) node[label=left:c$\times$] {} to[short, -*] (G2.in 2 |- G4);
  \draw (a |- G3.in 2) node[label=left:d$\times$] {} -- (G3.in 2);
  \draw (G2.in 1) node[label=e 1] {};
  \draw (G2.out) node[label=right:f 0/1] {};
  \draw (G3.out) node[label=right:g$\times$] {};
  \draw (G4.out) node[label=right:h$\times$] {};
  \draw[color=red, line width=0.75mm] (G4.in 1) -| (G2.out) |- (G1.out) |- (G1.in 2);
\end{circuitikz}
\captionof{figure}{Objective = (f, 0), inversion parity = even. Therefore, b = f = 0, e = 1.}
\end{center}

\begin{center} \begin{circuitikz}[line width=.7pt]
  \draw (0,0) node[nor port] (G1) {};
\draw (G1.out) -- ++(0,0) node[nand port, anchor=in 1] (G2) {};
\draw (G2.in 2) -- ++(0,-1) node[xnor port, anchor=in 1] (G3) {};
\draw (G2.out) |- ++(0.5,-0.5) node[nor port, anchor=in 1] (G4) {};
\draw (G3.out) |- (G4.in 2);

\draw (G1) node[left=6pt] {G1};
\draw (G2) node[left=6pt] {G2};
\draw (G3) node[left=6pt] {G3};
\draw (G4) node[left=6pt] {G4};

  \draw (G1.in 1) node[label=left:a 0] (a) {};
  \draw (G1.in 2) node[label=left:b 0] {};
  \draw (a |- G4) node[label=left:c 1] (c) {} to[short, -*] (G2.in 2 |- G4);
  \draw (a |- G3.in 2) node[label=left:d$\times$] {} -- (G3.in 2);
  \draw (G2.in 1) node[label=e 1] {};
  \draw (G2.out) node[label=right:f 0/1] {};
  \draw (G3.out) node[label=right:g$\times$] {};
  \draw (G4.out) node[label=right:h$\times$] {};
  \draw[color=red, line width=0.75mm] (G4.in 1) -| (G2.out) -| (G2.in 2) |- (c);
\end{circuitikz}
\captionof{figure}{Objective = (f, 0), inversion parity = odd. Therefore, c = f' = 1}
\end{center}

\begin{center} \begin{circuitikz}[line width=.7pt]
  \draw (0,0) node[nor port] (G1) {};
\draw (G1.out) -- ++(0,0) node[nand port, anchor=in 1] (G2) {};
\draw (G2.in 2) -- ++(0,-1) node[xnor port, anchor=in 1] (G3) {};
\draw (G2.out) |- ++(0.5,-0.5) node[nor port, anchor=in 1] (G4) {};
\draw (G3.out) |- (G4.in 2);

\draw (G1) node[left=6pt] {G1};
\draw (G2) node[left=6pt] {G2};
\draw (G3) node[left=6pt] {G3};
\draw (G4) node[left=6pt] {G4};

  \draw (G1.in 1) node[label=left:a 0] (a) {};
  \draw (G1.in 2) node[label=left:b 0] {};
  \draw (a |- G4) node[label=left:c 1] (c) {} to[short, -*] (G2.in 2 |- G4);
  \draw (a |- G3.in 2) node[label=left:d 0] (d) {} -- (G3.in 2);
  \draw (G2.in 1) node[label=e 1] {};
  \draw (G2.out) node[label=right:f 0/1] {};
  \draw (G3.out) node[label=right:g 1] {};
  \draw (G4.out) node[label=right:h 0/1] {};
  \draw[color=red, line width=0.75mm] (G4.in 2) -| (G3.out) |- (G3.in 2) |- (d);
\end{circuitikz}
\captionof{figure}{To propagate to h, g = 1. Objective = (g, 1), inversion parity = odd. Therefore, d = g' = 0. f-sa1 is now observable at h, thus complete.}
\end{center}

\textbf{Answer:} (a, b, c, d) = (0, 0, 1, 0).

