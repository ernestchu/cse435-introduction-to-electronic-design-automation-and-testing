\documentclass[12pt,answers]{exam}
\usepackage{fontspec,graphicx,circuitikz,amsmath,caption,tikz,ctable,multirow,array}
\usepackage[none]{hyphenat}
\setmainfont{Times New Roman}

\headheight 8pt \headsep 20pt \footskip 30pt
\textheight 9in \textwidth 6.5in
\oddsidemargin 0in \evensidemargin 0in
\topmargin -.35in

\begin{document}
\begin{center}
\LARGE CSE435 Introduction to EDA \& Testing - Spring 2022 \\
\Large Homework Assignment \#6 \\
\Large Shao-Hsuan Chu - B073040018 \\
\end{center}
\bigskip

About Path-Oriented Decision Making (PODEM), please answer the following questions according to Figure 1. \\
1-4 PODEM (See Figure 1)
\begin{questions}
  \question (15\%) Derive the effective test set for G5 SA0 by PODEM
  The defect level $\textbf{DL}$ can be obtained by $\textbf{DL} = 1 - \textbf{Y} ^ {1 - \textbf{T}}$, where $\textbf{Y}$ is the yield, indicating the manufacturing capability, and $\textbf{T}$ is the fault coverage, indicating the testing capability.
\begin{align*} 
0.001                             &\geq 1 - 0.9 ^ {1 - \textbf{T}} \\
0.9 ^ {1 - \textbf{T}}            &\geq 1 - 0.001 \\
0.9 ^ {1 - \textbf{T}}            &\geq 0.999 \\
\log_{0.9} 0.9 ^ {1 - \textbf{T}} &\leq \log_{0.9} 0.999 \\
1 - \textbf{T}                    &\leq 0.0095 \\
\textbf{T}                        &\geq 0.9905 = 99.05\%
\end{align*}
\textbf{Answer:} The fault coverage $\textbf{Y}$ must be at least 99.05\%

  \begin{solution}
    The defect level $\textbf{DL}$ can be obtained by $\textbf{DL} = 1 - \textbf{Y} ^ {1 - \textbf{T}}$, where $\textbf{Y}$ is the yield, indicating the manufacturing capability, and $\textbf{T}$ is the fault coverage, indicating the testing capability.
\begin{align*} 
0.001                             &\geq 1 - 0.9 ^ {1 - \textbf{T}} \\
0.9 ^ {1 - \textbf{T}}            &\geq 1 - 0.001 \\
0.9 ^ {1 - \textbf{T}}            &\geq 0.999 \\
\log_{0.9} 0.9 ^ {1 - \textbf{T}} &\leq \log_{0.9} 0.999 \\
1 - \textbf{T}                    &\leq 0.0095 \\
\textbf{T}                        &\geq 0.9905 = 99.05\%
\end{align*}
\textbf{Answer:} The fault coverage $\textbf{Y}$ must be at least 99.05\%

  \end{solution}

  \question (15\%) Derive the effective test set for G5 SA0 by PODEM with More Intelligent Backtracing
  \begin{solution}
    The test pattern can be given as the table below. Each row is a clock tick. Assume $\text{Q}_\text{n}$ = 0 at the initial tick. {\color{red} (Todo: How to make such assumption?)} For all the following ticks, $\text{Q}_\text{n}$ is the $\text{Q}_\text{n+1}$ from the previous tick.
\begin{center}
  \begin{tabular}{c | c | c}
    T & $\text{Q}_\text{n+1}$ & Functions \\
    \hline
    0 & 0 & hold 0 \\
    1 & 1 & toggle to 1 \\
    0 & 1 & hold 1 \\
    1 & 0 & toggle to 0 \\
  \end{tabular}
\end{center}
If output does not meet the expectation, a functional fault is detected in the F/F. One of the hold or toggle to 0/1 functions could not behave properly.

  \end{solution}

  \question (15\%) Derive the effective test set for G5 SA0 by PODEM with Unguided Backtracing
  \begin{solution}
    The test pattern can be given as the table below. Each row is a clock tick, and at the initial tick, $\text{Q}_\text{n}$ is in the don't-care condition. For all the following ticks, $\text{Q}_\text{n}$ is the $\text{Q}_\text{n+1}$ from the previous tick.
\begin{center}
  \begin{tabular}{c c | c | c}
    J & K & $\text{Q}_\text{n+1}$ & Functions \\
    \hline
    0 & 1 & 0 & set 0 \\
    0 & 0 & 0 & hold 0 \\
    1 & 0 & 1 & set 1 \\
    0 & 0 & 1 & hold 1 \\
    1 & 1 & 0 & toggle to 0 \\
    1 & 1 & 1 & toggle to 1 \\
  \end{tabular}
\end{center}
If output does not meet the expectation, a functional fault is detected in the F/F. One of the set, hold or toggle to 0/1 functions could not behave properly.

  \end{solution}

  \question (15\%) Please compare the above (1), (2), (3). [Chapter 10, slides 7-23]
  \begin{parts}
    \part (7\%) What are the differences between (1) and (2)/(3)?
    \begin{solution}
      In the figure below, the red circles indicate the missing crosspoint faults, and the upward arrows indicate the s-a-1 faults. The missing crosspoint in the AND plane causes growth fault, i.e., the lines having crosspoint with C will no longer depend on C.
\begin{center}
  \begin{minipage}{0.4\linewidth}
    \raggedleft
    \begin{circuitikz}[line width=.7pt]
      \ctikzset{logic ports=ieee}
    
      % A & A negate
\draw (0,0) node[label=A] (A) {} -- ++(0,-4.5);
\draw (0.5,-1) node[not port, rotate=-90, scale=0.4, circuitikz/ieeestd ports/not radius=.25] (A_neg) {};
\draw (A_neg.in) |- ++(-0.5,0.25) to[short, *-] ++(0,0);
\draw (A_neg.out) -- (0.5,-4.5);

% B & B negate
\draw (1,0) node[label=B] (B) {} -- ++(0,-4.5);
\draw (1.5,-1) node[not port, rotate=-90, scale=0.4, circuitikz/ieeestd ports/not radius=.25] (B_neg) {};
\draw (B_neg.in) |- ++(-0.5,0.25) to[short, *-] ++(0,0);
\draw (B_neg.out) -- (1.5,-4.5);

% C & C negate
\draw (2,0) node[label=C] (C) {} -- ++(0,-4.5);
\draw (2.5,-1) node[not port, rotate=-90, scale=0.4, circuitikz/ieeestd ports/not radius=.25] (C_neg) {};
\draw (C_neg.in) |- ++(-0.5,0.25) to[short, *-] ++(0,0);
\draw (C_neg.out) -- (2.5,-4.5);

% f1 & f2
\draw (4.0,0) node[label=f1] (f1) {} -- ++(0,-4.5);
\draw (4.5,0) node[label=f2] (f2) {} -- ++(0,-4.5);

% horizontal lines
\draw (-0.5,-1.5) -- ++(5.5,0);
\draw (-0.5,-2.0) -- ++(5.5,0);
\draw (-0.5,-2.5) -- ++(5.5,0);
\draw (-0.5,-3.0) -- ++(5.5,0);
\draw (-0.5,-3.5) -- ++(5.5,0);
\draw (-0.5,-4.0) -- ++(5.5,0);

% and plane
% AB
\draw (0,-1.5) to[short, *-*] ++(1,0);
% BC
\draw (1,-2.0) to[short, *-*] ++(1,0);
% CA
\draw (0,-2.5) to[short, *-*] ++(2,0);
% A_negB_neg
\draw (0.5,-3.0) to[short, *-*] ++(1,0);
% B_negC_neg
\draw (1.5,-3.5) to[short, *-*] ++(1,0);
% C_negA_neg
\draw (0.5,-4.0) to[short, *-*] ++(2,0);

    
      % f1 or plane
      \draw (4,-1.5) node[label=center:$\times$] {};
      \draw (4,-2.0) node[label=center:$\times$] {};
      \draw (4,-2.5) node[label=center:$\times$] {};
      \draw (4,-3.0) node[label=center:$\times$] {};
      \draw (4,-3.5) node[label=center:$\times$] {};
      \draw (4,-4.0) node[label=center:$\times$] {};
    
      % f2 or plane
      \draw (4.5,-1.5) node[label=center:$\times$] {};
      \draw (4.5,-2.0) node[label=center:$\times$] {};
      \draw (4.5,-2.5) node[label=center:$\times$] {};
    
      % fault
      \draw[color=red] (2,-1.5) circle[radius=6pt] {};
      \draw[color=red] (2,-2.0) circle[radius=6pt] {};
      \draw[color=red] (2,-2.5) circle[radius=6pt] {};
      \draw[color=red] (2,-3.0) circle[radius=6pt] {};
      \draw[color=red] (2,-3.5) circle[radius=6pt] {};
      \draw[color=red] (2,-4.0) circle[radius=6pt] {};
      
    \end{circuitikz}
  \end{minipage}
  \hfill
  \begin{minipage}{0.5\linewidth}
    \begin{circuitikz}[line width=.7pt]
      \ctikzset{logic ports=ieee}

      % A & A negate
\draw (0,0) node[label=A] (A) {} -- ++(0,-4.5);
\draw (0.5,-1) node[not port, rotate=-90, scale=0.4, circuitikz/ieeestd ports/not radius=.25] (A_neg) {};
\draw (A_neg.in) |- ++(-0.5,0.25) to[short, *-] ++(0,0);
\draw (A_neg.out) -- (0.5,-4.5);

% B & B negate
\draw (1,0) node[label=B] (B) {} -- ++(0,-4.5);
\draw (1.5,-1) node[not port, rotate=-90, scale=0.4, circuitikz/ieeestd ports/not radius=.25] (B_neg) {};
\draw (B_neg.in) |- ++(-0.5,0.25) to[short, *-] ++(0,0);
\draw (B_neg.out) -- (1.5,-4.5);

% C & C negate
\draw (2,0) node[label=C] (C) {} -- ++(0,-4.5);
\draw (2.5,-1) node[not port, rotate=-90, scale=0.4, circuitikz/ieeestd ports/not radius=.25] (C_neg) {};
\draw (C_neg.in) |- ++(-0.5,0.25) to[short, *-] ++(0,0);
\draw (C_neg.out) -- (2.5,-4.5);

% horizontal lines
\draw (-0.5,-1.5) -- ++(3.25,0) -- ++(0,0) node[and port, anchor=in 1, scale=0.35715] (p1) {};
\draw (-0.5,-1.7) -- ++(3.25,0) -- (p1.in 2);
\draw (-0.5,-2.0) -- ++(3.25,0) -- ++(0,0) node[and port, anchor=in 1, scale=0.35715] (p2) {};
\draw (-0.5,-2.2) -- ++(3.25,0);
\draw (-0.5,-2.5) -- ++(3.25,0) -- ++(0,0) node[and port, anchor=in 1, scale=0.35715] (p3) {};
\draw (-0.5,-2.7) -- ++(3.25,0);
\draw (-0.5,-3.0) -- ++(3.25,0) -- ++(0,0) node[and port, anchor=in 1, scale=0.35715] (p4) {};
\draw (-0.5,-3.2) -- ++(3.25,0);
\draw (-0.5,-3.5) -- ++(3.25,0) -- ++(0,0) node[and port, anchor=in 1, scale=0.35715] (p5) {};
\draw (-0.5,-3.7) -- ++(3.25,0);
\draw (-0.5,-4.0) -- ++(3.25,0) -- ++(0,0) node[and port, anchor=in 1, scale=0.35715] (p6) {};
\draw (-0.5,-4.2) -- ++(3.25,0);

% and plane
% AB
\draw (0,-1.5)  to[short, *-] ++(0,0);
\draw (1,-1.7) to[short, *-] ++(0,0);
% BC
\draw (1,-2.0)  to[short, *-] ++(0,0);
\draw (2,-2.2) to[short, *-] ++(0,0);
% CA
\draw (0,-2.5)  to[short, *-] ++(0,0);
\draw (2,-2.7) to[short, *-] ++(0,0);
% A_negB_neg
\draw (0.5,-3.0)  to[short, *-] ++(0,0);
\draw (1.5,-3.2) to[short, *-] ++(0,0);
% B_negC_neg
\draw (1.5,-3.5)  to[short, *-] ++(0,0);
\draw (2.5,-3.7) to[short, *-] ++(0,0);
% C_negA_neg
\draw (0.5,-4.0)  to[short, *-] ++(0,0);
\draw (2.5,-4.2) to[short, *-] ++(0,0);

    
      % f1
      \draw (5,-2.85) node[or port, scale=0.5, number inputs=6] (f1) {};
      \draw (p1.out) -- ++(0.2,0) |- (f1.in 1);
      \draw (p2.out) -- ++(0.1,0) |- (f1.in 2);
      \draw (p3.out) |- (f1.in 3);
      \draw (p4.out) |- (f1.in 4);
      \draw (p5.out) -- ++(0.1,0) |- (f1.in 5);
      \draw (p6.out) -- ++(0.2,0) |- (f1.in 6);
      \draw (f1.out) -- ++(0,0) node[label=right:f1] {};

      % f2
      \draw (5,-2) node[or port, scale=0.5, number inputs=3] (f2) {};

      \draw (f1.in 1) to[short,*-] (f2.in 3);

      \draw (f2.in 2) ++(-0.2,0) node (f2_in2_shifted) {} -- (f2.in 2);
      \draw (f1.in 2) ++(-0.2,0) to[short,*-] (f2_in2_shifted.center);

      \draw (f2.in 1) ++(-0.4,0) node (f2_in1_shifted) {} -- (f2.in 1);
      \draw (f1.in 3) ++(-0.4,0) to[short,*-] (f2_in1_shifted.center);

      \draw (f2.out) -- ++(0,0) node[label=right:f2] {};
    
      % fault
      \draw[color=red, -latexslim] (p2.in 2) ++(-0.05,-0.15) -- ++(0,0.3);
      \draw[color=red, -latexslim] (p3.in 2) ++(-0.05,-0.15) -- ++(0,0.3);
      \draw[color=red] (3.25,-1.5) node[label=s-a-1] {};
    \end{circuitikz}
  \end{minipage}
\end{center}

    \end{solution}

    \part (8\%) What are the differences between (2) and (3)?
    \begin{solution}
      
(2) follows the guidance of the controllability measures. It encounters zero conflict and thus performs no backtracking.

(3) intentionally disobeys the guidance of the controllability measures. It encounters two conflict and thus performs two backtracking. The empirical result shows the heuristic does reduce the possibility of encountering a conflict.

    \end{solution}
  \end{parts}

  \question (15\%) What are the difference between Backtracking and Backtracing? Please show by example.
  \begin{solution}
    
A backtracing procedure maps the objective into a PI assignment that is likely to contribute to achieve the objective, while a backtracking procedure goes back to the last decision when encountering a conflict.

Take the circuit above as an example, we backtrace to the primary inputs until they satisfy the given objective in the simulation. After each backtracing, if the assignment to the primary input conflicts with the objective, we undo the assignment and try another test pattern.

  \end{solution}

  \question (25\%) FAN (See Figure 2)
  \begin{center}
\begin{circuitikz}[line width=.7pt]
  \draw (0,0) node[nor port] (G1) {};
  \draw (G1.out) -- ++(0,0) node[nand port, anchor=in 1] (G2) {};
  \draw (G2.in 2) -- ++(0,-1) node[xnor port, anchor=in 1] (G3) {};
  \draw (G2.out) |- ++(0.5,-0.5) node[nor port, anchor=in 1] (G4) {};
  \draw (G3.out) |- (G4.in 2);

  \draw (G1) node[left=6pt] {G1};
  \draw (G2) node[left=6pt] {G2};
  \draw (G3) node[left=6pt] {G3};
  \draw (G4) node[left=6pt] {G4};
  \draw (G1.in 1) node[label=left:a] (a) {};
  \draw (G1.in 2) node[label=left:b] {};
  \draw (a |- G4) node[label=left:c] {} to[short, -*] (G2.in 2 |- G4);
  \draw (a |- G3.in 2) node[label=left:d] {} -- (G3.in 2);
  \draw (G2.in 1) node[label=e] {};
  \draw (G2.out) node[label=right:f] {};
  \draw (G3.out) node[label=right:g] {};
  \draw (G4.out) node[label=right:h] {};
\end{circuitikz}
\captionof{figure}{}
\end{center}

  \begin{parts}
    \part (15\%) Generate a test for the fault f-sa1 by using FAN algorithm.
    \begin{solution}
      
\begin{enumerate}
  \item Identify headlines: c, d, e.
  \item Activate f-sa1. Assign D' to f.
  \item Backward implication: c = e = 1, a = b = 0.
  \item Since G4 is an unique D-frontier, perform unique sensitization. Assign 0 to g.
  \item Backward implication: d = 0.
  \item Forward implication: h = D. The fault has reached the primary output.
  \item No justification needed. Test generated, (a, b, c, d) = (0, 0, 1, 0).
\end{enumerate}

\textbf{Answer}: (a, b, c, d) = (0, 0, 1, 0).

    \end{solution}

    \part (10\%) What are the differences between PODEM and FAN?
    \begin{solution}
      
\begin{center}
  \begin{tabular}{p{2cm}|p{5cm}p{.5cm}p{5cm}}
    \specialrule{.1em}{.05em}{.05em} 
    & \textbf{PODEM} &&	\textbf{FAN} \\
    \hline
    Assignment location & primary inputs && headlines (internal lines) \\
    \hline
    Conflict location & primary inputs && headlines (internal lines) \\
    \hline
    Justification & no && headlines to the primary inputs \\
    \hline
    Implication & forward only && forward \& backward \\
    \hline
    Backtracing & single backtracing to the primary input (DFS) && multiple parallel backtracing to the headlines (BFS) \\
    \hline
    Unique sensitization & no && yes \\
    \specialrule{.1em}{.05em}{.05em} 
  \end{tabular}
  \captionof{table}{The differences between PODEM and FAN.}
\end{center}

    \end{solution}
  \end{parts}

\end{questions}
\end{document}

