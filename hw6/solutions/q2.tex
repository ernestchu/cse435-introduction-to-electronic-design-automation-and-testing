
\begin{enumerate}
  \item Activate G5 SA0. Initial objective: (G5, D).
  \item {
    \begin{enumerate}
      \item Since G5 is an imply gate, backtrace to the hardest unknown input, G1, with CY1(G1) = 0.25. With an even inversion parity, intermediate objective: (G1, 1).
      \item Since G1 is an imply gate, keep backtrace to the hardest unknown input, A, (CY1(A) = CY1(B) = 0.5, so we may choose either one). With an even inversion parity, intermediate objective: (A, 1).
      \item Since A is not a gate, stop backtracing. Assign 1 to A. Simulation, objective not achieved. Implication: G3 = 0.
    \end{enumerate}
  }
  \item {
    \begin{enumerate}
      \item Since G5 is an imply gate, backtrace to the hardest unknown input, G1, with CY1(G1) = 0.25. With an even inversion parity, intermediate objective: (G1, 1).
      \item Since G1 is an imply gate, keep backtrace to the hardest unknown input, B. With an even inversion parity, intermediate objective: (B, 1).
      \item Since B is not a gate, stop backtracing. Assign 1 to B. Simulation, objective not achieved. Implication: G1 = 1, G6 = 0.
    \end{enumerate}
  }
  \item {
    \begin{enumerate}
      \item Since G5 is an imply gate, backtrace to the hardest unknown input, G4. With an odd inversion parity, intermediate objective: (G4, 0).
      \item Since G4 is an imply gate, keep backtrace to the hardest unknown input, G2. With an even inversion parity, intermediate objective: (G2, 0).
      \item Since G2 is an imply gate, keep backtrace to the hardest unknown input, C. With an even inversion parity, intermediate objective: (C, 0).
      \item Since C is not a gate, stop backtracing. Assign 0 to C. Simulation, objective achieved. Implication: G2 = 0, G4 = 0, G5 = D, G7 = D. The fault has reached the primary output.
    \end{enumerate}
  }
  \item Test generated, (A, B, C) = (1, 1, 0).

\end{enumerate}

\textbf{Answer:} (A, B, C) = (1, 1, 0).

